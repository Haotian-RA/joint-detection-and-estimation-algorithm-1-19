\section{Introduction}%
\label{sec:introduction}

% In digital communication systems, information is commonly transmitted in time-multiplexed bursts.
% % Examples include time-slotted random access systems.
% Each active user transmits information to the receiver
% in the same frequency band and in non-overlapping time intervals~\cite{Falconer_95}. 
% Our methods are pertinent for all TDMA systems where the arrival time
% of the burst and the carrier frequency and phase are
% not perfectly known.
% Obvious examples include random access channels, including those in cellular
% systems. Even when a coarse timing structure is imposed, as in slotted
% TDMA systems, the precise arrival time of a signal may not be known;
% this is generally true for the uplink of cellular-like systems. 
% In such cases, the methods presented herein are still relevant  by
% restricting the search for the start of the burst to periods near the
% start of a TDMA~slot.

Protecting information in contested and congested wireless
communication environments requires transmissions that are difficult
to intecept or even detect (LPI/LPD) and are robust against
intentional or accidental interference (AJ).
A key measure to achieve protection against detection is to rely on
signals with power spectral densities that are well below the noise or
interference floor~\cite{Yan_19}.
With appropriate processing gains built into the transmitted waveform, signals can be
recovered reliably even in the presence of strong noise and
interference.

A fundamental prerequisite for leveraging such processing gains and
enable successful coherent demodulation is that the receiver can
detect the beginning of the data stream and estimate accurately the phase and frequency offset of the carrier.
It is worth emphasizing that time and carrier synchronization are
\emph{coupled problems}, especially at the low SNR that secure
waveforms operate at:
coherent methods for detecting the signal require accurate frequency and phase estimates 
while data-aided frequency and phase estimation requires that the
location of the training sequence is available.
While not addressed in detail in this paper, coherent combing of
spatial streams (MIMO) also relies on accurate synchronization.
Thus, joint signal detection and carrier synchronization algorithms
play a vital role in protected communication system.

Clearly, the signal acquisition problem has been considered widely. 
Classical work on carrier synchronization can be found, e.g., in a
review by  Morelli
and Mengali~\cite{Morelli_Mengali_98}. 
While their review assumes that time synchronization (detection) has been
accomplished, e.g., by noncoherent means, 
work in~\cite{kay_89,Fitz_94,Luise_Reggiannini_95}
is still relevant to results in this paper.
It was argued above that at low SNR, time and carrier synchronization
are not separable and must be performed jointly~\cite{purushothaman_16,kim_17}.
To this end, 
a sequential detector is derived from hypothesis testing principles
for varied length of frame in this paper.
% a sequential detector
% is  derived from hypothesis testing principles~\cite{Ramakrishnan_10,Chiani_06,Liang_15}.
As the detector is lacking carrier related parameters to construct 
the likelihood ratio test (LRT), a generalized likelihood ratio test (GLRT)
is employed to cope with the lack of information, e.g.,~\cite{liang_15}.

Computational complexity is a major concern with the practical
realization of a synchronization system outlined above and operating at low SNR~\cite{murin_16,wang_21}.
While the sequential detector searches for the start of the
transmission, carrier-related parameters must be estimated for each
hypothesized start time. In this paper,
we propose a family of joint detection and estimation algorithm 
for the complete signal acquisition process
that emphasize computational complexity while maintaining near-optimal performance at very low SNR (negative SNR).
The practicality of our algorithms is demonstrated by implementing
them on a standard SDR platform. To our best knowledge, there is no previous work on discussing 
the signal acquisition process in such a comprehensive way, especially at negative SNRs, which
is very meanful for protected communication system.



% ; in our experiments, we can sustain
% above \SI{10}{\mega\hertz} sample rates while continuously searching
% for the presence of the preamble signal. -> I think we can talk about it in abstract.
