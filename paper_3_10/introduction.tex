\section{Introduction}%
\label{sec:introduction}

In digital communication systems, information is commonly transmitted in time-multiplexed bursts.
% Examples include time-slotted random access systems.
Each active user transmits information to the receiver
in the same frequency band and in non-overlapping time intervals~\cite{Falconer_95}. 
Our methods are pertinent for all TDMA systems where the arrival time
of the burst and the carrier frequency and phase are
not perfectly known.
Obvious examples include random access channels, including those in cellular
systems. Even when a coarse timing structure is imposed, as in slotted
TDMA systems, the precise arrival time of a signal may not be known;
this is generally true for the uplink of cellular-like systems. 
In such cases, the methods presented herein are still relevant  by
restricting the search for the start of the burst to periods near the
start of a TDMA~slot. 

A fundamental prerequisite for successful coherent demodulation is that the receiver can
detect the beginning of the data stream and estimate accurately the phase and frequency offset of the carrier.
It is worth emphasizing that time and carrier synchronization are \emph{coupled problems}, especially at low SNRs:
coherent methods for detecting the signal require accurate frequency and phase estimates 
while data-aided frequency and phase estimation requires that the location of the training sequence is available.
Thus, joint signal detection and carrier synchronization algorithms play a vital role in any communication system.

Clearly, the signal acquisition problem has been considered widely. 
In the late 90's, Morelli and Mengali~\cite{Morelli_Mengali_98} presented a tutorial review
of the carrier synchronization field comparing such characteristics as estimation accuracy, range,
and computational complexity of available techniques.
The work by~\cite{kay_89,Fitz_94,Luise_Reggiannini_95}
is most closely related to results in this paper.
For signal detection and
particularly, sequential detection, the detector
is commonly derived from hypothesis testing principles~\cite{Ramakrishnan_10,Chiani_06,Liang_15}.
When the detector is lacking parameters to construct 
likelihood ratio test (LRT), generalized likelihood ratio tests (GLRT)
are employed to cope with the lack of information, e.g.,~\cite{Chiani_06}. 

In this paper, we propose a family of joint detection and estimation algorithm 
for the complete signal acquisition process.
We emphasize computational complexity while maintaining near-optimal performance.
The practicality of our algorithms is demonstrated by implementing
them on a standard SDR platform; in our experiments we can sustain
nearly \SI{10}{\mega\hertz} sample rates while continuously searching
for the presence of the preamble signal.
