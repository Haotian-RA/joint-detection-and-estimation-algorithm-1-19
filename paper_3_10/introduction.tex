\section{Introduction}%
\label{sec:introduction}


Protecting information in contested and congested wireless
communication environments requires transmissions that are difficult
to intercept or even detect (LPI/LPD) and that are robust against
intentional or accidental interference (AJ).
A key measure to achieve robust protection against detection is to rely on
signals with power spectral densities that are well below the noise or
interference floor~\cite{Yan_19}.
With appropriate processing gains built into the transmitted waveform, signals can be
recovered reliably even in the presence of strong noise and
interference.

A fundamental prerequisite for leveraging such processing gains and
for enabling successful coherent demodulation is that the receiver can
detect the beginning of the data stream and
can estimate accurately the phase and frequency offset of the carrier.
It is worth emphasizing that time and carrier synchronization are
\emph{coupled problems}, especially at the low SNR that secure
waveforms operate at:
coherent methods for detecting the signal at low SNR require accurate frequency and phase estimates 
while data-aided frequency and phase estimation requires that the
location of the training sequence is available.
While not addressed in this paper, coherent combing of
spatial streams (MIMO) also relies on accurate synchronization.
Thus, joint signal detection and carrier synchronization algorithms
play a vital role in protected communication system.

Clearly, the signal acquisition problem has been considered widely. 
Classical work on carrier synchronization can be found, e.g., in a
review by  Morelli
and Mengali~\cite{Morelli_Mengali_98}. 
While their review assumes that time synchronization (detection) has been
accomplished, e.g., by noncoherent means, 
work in~\cite{kay_89,Fitz_94,Luise_Reggiannini_95}
is still relevant to results in this paper.
It was argued above that at low SNR, time and carrier synchronization
are not separable and must be performed jointly; see also~\cite{purushothaman_16,kim_17}.
To this end, 
a sequential detector is derived from hypothesis testing principles.
As the detector is lacking carrier related parameters to construct 
the likelihood ratio test (LRT), a generalized likelihood ratio test (GLRT)
can be employed to cope with the lack of information, e.g.,~\cite{liang_15}.

Computational complexity is a major concern with the practical
realization of a synchronization system outlined above and operating at low SNR~\cite{murin_16,wang_21}.
While the sequential detector searches for the start of the
transmission, carrier-related parameters must be estimated for each
hypothesized start time. In this paper,
we propose a family of joint detection and estimation algorithm 
for the complete signal acquisition process
that emphasize computational complexity while maintaining near-optimal
performance at very low SNR.
The practicality of our algorithms is demonstrated by implementing
them on a standard SDR platform.
To the best of our knowledge, this paper's comprehensive treatment of the signal
acquisition problem at low SNR has not been presented previously.
The combination of rigorous algorithm development, careful analysis,
and implementation provide a unique contribution to the problem of
protected communications.



% ; in our experiments, we can sustain
% above \SI{10}{\mega\hertz} sample rates while continuously searching
% for the presence of the preamble signal. -> I think we can talk about it in abstract.
