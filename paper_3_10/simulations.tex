\section{Simulation Results}%
\label{sec:simulations}

\begin{figure}[t]
    \centerline{\includegraphics[width=3.25in]{accuracy_NM_SL.png}}
    \caption{Accuracy of the NM estimator and single-lag estimator ($L_0=32$)}%
    \label{fig:accuracy_NM_SL}
    \end{figure}

\begin{figure}[t]
    \centerline{\includegraphics[width=3.25in]{accuracy_NM_SL_traditional.png}}
    \caption{Accuracy of SL, NM and traditional estimators ($L_0=32$, $M\delta=0.01$)}%
    \label{fig:accuracy_NM_SL_traditional}
    \end{figure}

\begin{figure}[t]
    \centerline{\includegraphics[width=3.25in]{ROC_new.png}}
    \caption{Receiver operating characteristics (ROC) of the sequential detector}%
    \label{fig:Receiver operating characteristics}
    \end{figure}

In the simulation section, we reverse the order of discussion by first showing 
the accuracy of estimators in carrier synchronization and then showing some results of sequential detection since
the GLRT based detector in~\eqref{eq:generalized_corr} relies on the accuracy of 
the SL estimator.
The symbol sequence of the preamble is chosen as a Gold sequence 
with good autocorrelation properties and
modulated by a QPSK alphabet.
The pulse is chosen a
Square-Root Raised Cosine (SRRC) pulse with roll-off equal to 0.5.
The normalized frequency offset $\delta$ is intentionally set to be in
the safe estimation range for all estimators for simulation purposes. 

\subsection{Simulation Results for Estimation}%

% \begin{figure}[t]
%     \centerline{\includegraphics[width=3.4in]{generalized_correlation_p_plus_delta.png}}
%     \caption{Performance of sequential detector when the preamble is pulse shaped}
%     \label{fig:Generalized correlation}
%     \end{figure}

Figure~\ref{fig:accuracy_NM_SL} illustrates the accuracy of single-lag (SL) and the NM estimator.
Comparing the two curves for SL estimators with $v{=}1$ and $v{=}16$, 
we see the length-$16$ partial integration
improves the accuracy by providing an approximate
\SI{2.5}{\dB}~performance gain at negative SNRs
(near $\text{SNR}=\SI{-5}{\dB}$). This is consistent
with~\eqref{eq:relative_processing_gain}.

Moreover, at high SNR the SL estimator with $v{=}1$ approaches the lower bound~\eqref{eq:lower_bound_single_lag_high_snr} (with $v=1$)
while the SL estimator with $v{=}16$ does not. 
The gap is due to the Dirichlet function 
with respect to $v$ and $|\delta|$.
For the same reason, the accuracy of the SL estimator with $v=16$ has
a better accuracy at all SNRs when the frequency offset $\delta$ is small.

The SL estimators do not approach the CRVB~\cite{Gini_98} while the NM
estimator does at high SNR. 
We also see that the NM estimator achieves improved accuracy at lower SNR
when the SL estimator with $v=16$ is used as the starting point. 
In contrast, at all negative SNRs the NM estimator based on SL estimation with $v=1$ has
worse accuracy than SL estimation alone.
This is because the accuracy of the SL estimator is not sufficient to
provide a consistently good starting point and the Newton iteration converges occasionally to 
other local minima away from the true frequency offset.

Figure~\ref{fig:accuracy_NM_SL_traditional} compares the accuracy of our proposed estimators
and the estimators in~\cite{kay_89,Fitz_94,Luise_Reggiannini_95}.
When the 
frequency offset is small, our NM estimator has a slightly better accuracy than other estimators at moderate SNRs.
In general, our family of  SL estimators with different block lengths $v$
are very competitive across all SNR considered here while maintaining
low computational complexity that allows for real-time application.

\subsection{Simulation Results for Detection}

% Figure~\ref{fig:Generalized correlation} shows the performance of sequential detector
% in~\eqref{eq:generalized_corr} at each received signal windows. Note, because of pulse sha-ping,
% the autocorrelation property of the preamble sequence is decreased, which results
% the correlations at adjacent windows near the position of the preamble decay slow and thus make much challenges
% to choose threshold to make correct dicision.

% The solution is to adjust the detection algorithm by finding the local maximum of the correlation
% instead of just comparing the correlation with the threshold at each window. Note, when count for the ratio of false alarm and detection (for determining the threshold),
% the two positions should be counted as latter.

Figure~\ref{fig:Receiver operating characteristics} shows the receiver
operating characteristics (ROC) of the detection algorithm. 
The better accuracy of SL estimation with partial integration also
improves the detection performance  at low SNR.
For example,
at $-2$ dB SNR, with $\gamma=0.56$, the gap in the false alarm
probability $P_{FA}$ is only $0.1\%$
but the detection probability $P_{D}$ with SL estimation with $v{=}16$
is $5\%$ larger than $P_D$ achieved with SL estimation with $v{=}1$. 
The figure also shows that the detector does not work well at $-4$ dB SNR if only $32$ symbols of preamble are used;  
The performance is significantly improved by doubling the length of
the preamble.
