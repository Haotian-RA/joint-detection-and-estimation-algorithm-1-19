\section{Frequency and Phase Estimation}%
\label{sec:freq_est}   

In this section, we discuss the carrier synchronization problem by assuming the known start time of the preamble in the received sequence.
For estimating the frequency offset $\delta$ and phasor $\xi=Ae^{j\phi}$, the maximum likelihood estimation (MLE) of the parameters in~\eqref{eq:model} is given by

\begin{equation}
    \label{eq:ML_f_xi}
      \hat{\delta},\hat{\xi}=\min_{\delta,\xi=Ae^{j\phi}}\sum_{n=0}^{N-1}|r_n-s_n\xi e^{j2\pi\delta n}|^{2}.
    \end{equation}
A closed form for $\hat{\xi}$ is readily derived by taking the Wirtinger derivative with $\xi$ and setting it equal to zero,

\begin{equation}
    \label{eq:opt_xi}
    \hat{\xi}=\frac{\sum_{n=0}^{N-1}{r_{n}s_n^{*}e^{-j2\pi\hat{\delta} n}}}{\sum_{n=0}^{N-1}|s_{n}|^2},
  \end{equation}
so that $\hat{\xi}$ relies on the frequency estimate $\hat{\delta}$, and $\hat{\phi}=\arg\{\hat{\xi}\}$. 
% Moreover, by plugging~\eqref{eq:opt_xi} into~\eqref{eq:generalized_corr}, the GLRT based detector can be computed more
% efficiently,

% \begin{equation}
%     \label{eq:simplified_GLRT_detector}
%     \rho(p{+}\Delta) \approx
%     \frac{|\tilde{\xi}_{p+\Delta}|}
%     {||\bm{r}_{p+\Delta}||\cdot||\bm{s}||} \LRT{H_1}{H_0} \gamma
%   \end{equation}
% where $\tilde{\xi}_{p+\Delta}=\sum_{n=0}^{N-1}{r_{n+p+\Delta}s_n^{*}e^{-j2\pi\hat{\delta}_{p+\Delta} n}}$ represents 
% the cross-correlation (the numerator) of the phasor estimate $\hat{\xi}$ in~\eqref{eq:opt_xi} at position $p{+}\Delta$, and $||\bm{s}||$ is the Euclidean norm of the preamble.

A necessary condition for the frequency offset estimate $\hat{\delta}$, denoted as $J(\hat{\delta})$,
is obtained similarly by taking the derivative of~\eqref{eq:ML_f_xi} with respect to
$\delta$ and setting it equal to zero. Skipping all intermediate derivation steps for brevity's sake,
it yields

\begin{equation}
    \label{eq:necessary condition for delta}
    J(\hat{\delta}) = \Im\bigg\{\sum_{k=1}^{N-1}{\sum_{m=k}^{N-1}{kr_{m-k}r_m^{*}s_{m-k}^{*}s_m}e^{j2\pi\hat{\delta}k}}\bigg\}=0.
    \end{equation}
Note, there are a number of local minima of~\eqref{eq:ML_f_xi} also satisfying
the necessary condition for $\hat{\delta}$ of~\eqref{eq:necessary condition for delta} in addition to the absolute minimum
(the exact solution of MLE). In~\cite{Luise_Reggiannini_95} and~\cite{Fitz_94}, the "false minima" are avoided
by appropriately restricting the operating range of the estimator. Specifically,
instead of calculating the sample autocorrelation functions for entire lag $k\in[1,N-1]$, 
they truncate~\eqref{eq:necessary condition for delta} by only considering the lag autocorrelation functions for $k\in[1,J]$,
where $J \ll N{-}1$. This is due to when $k$ is large, the number of autocorrelation functions is small for averaging,
so that~\eqref{eq:necessary condition for delta} produces a poor estimate for $\hat{\delta}$. 

Moreover, note that the estimator $\hat{\delta}$ in~\eqref{eq:necessary condition for delta} has no closed-form solution.
In~\cite{Luise_Reggiannini_95}, the necessary condition is approximated by replacing the exponential with its Taylor series expansion.
In~\cite{Fitz_94}, an approximate solution is obtained via Euler's identity for large $N$.
Both L\&R~\cite{Luise_Reggiannini_95} and Fitz~\cite{Fitz_94} estimators have computational complexity $O(N^2)$ 
reflecting the double summation. In~\cite{kay_89}, the Kay estimator reduces the complexity from $O(N^2)$ to $O(N)$ by only computing~\eqref{eq:necessary condition for delta} at lag $k=1$.
However, it suffers a bad accuracy at low SNRs.

In this paper, we propose a family of alternative solutions to~\eqref{eq:necessary condition for delta}.
A coarse solution with $O(N)$ complexity is used for operating at high sample rate during the sequential detection;
It prioritizes low complexity at the expense of some loss of accuracy. A second fine solution is used to improve
the estimation accuracy at moderate complexity for coherent demodulation once the preamble has been detected. 

\subsection{Coarse Solution: Single-Lag Estimator with Length-v Partial Correlating}

The first estimator is rooted in the insight that at high SNR, every lag $k$ in~\eqref{eq:necessary condition for delta}
can be used to approximate the frequency estimate $\delta$. By plugging $r_m\approx s_m\xi e^{j2\pi\delta m}$,
\eqref{eq:necessary condition for delta} is expanded to

\begin{equation}
  \label{eq:delta_extens_no_noise}
  \Im\bigg\{A^2\sum_{k=1}^{N-1}\sum_{m=k}^{N-1}k|s_{m-k}|^2|s_m|^2e^{j2\pi (\hat{\delta}-\delta)k}\bigg\}=0.
  \end{equation}
Note, the inner summation in~\eqref{eq:delta_extens_no_noise} is purely real for every lag $k$ if $\hat{\delta}=\delta$.
This suggests that an unbiased estimate of $\hat{\delta}$ can be obtained by using only a single lag $k$. The approach yields
a closed-form solution for $\hat{\delta}$. 
Next, we will show that by using a so-called length-$v$ coherent integrator,
the single-lag estimator also gains a good accuracy at low SNRs.

\subsubsection{Closed-form expression}
The single-lag (SL) estimator with length-$v$ partial correlating is given by

\begin{equation}
  \label{eq:single_lag_estimator_w_partial_corr}
  \hat{\delta}_{SL}^{(v)}(k_v)=-\frac{\arg\big\{\sum_{l=k_v}^{N/v-1}\digamma_l^*\digamma_{l-k_v}\big\}}{2\pi k_vv},
\end{equation}
where $\digamma$ denotes the coherent integrator, and

\begin{equation}
  \label{eq:coherent_integrator}
  \digamma_l=\sum_{n=lv}^{(l+1)v-1}r_ns_n^*, \quad \text{for}~l=0,1,\ldots,N/v{-}1.
\end{equation}
In~\eqref{eq:coherent_integrator}, $v$ is the number of coherent correlations for averaging in each coherent integrator. 
Normally, $v$ is set to be a factor of $N$ to include all the sample instants.
In~\eqref{eq:single_lag_estimator_w_partial_corr}, $k_v$ denotes the distance between
two coherent integrators for calculating the frequency estimate from non-coherent sample instants;
$k_v=\lfloor k/v \rfloor$, where $\lfloor \cdot \rfloor$ is the floor operation. 
Particularly, when $v{=}1$, meaning no partial correlating is used,~\eqref{eq:single_lag_estimator_w_partial_corr} reduces to

\begin{equation}
  \label{eq:single_lag_estimator_wout_partial_corr}
  \hat{\delta}_{SL}^{(1)}(k)=-\frac{\arg\big\{\sum_{m=k}^{N-1}r_{m-k}r_m^*s_{m-k}^*s_m\big\}}{2\pi k},
\end{equation}
which is exactly the closed-form solution for $\hat{\delta}$ to~\eqref{eq:delta_extens_no_noise} with single lag $k$.

\subsubsection{Performance of single-lag estimator}

For evaluating the performance of the SL estimator,
we first look at the probability density function (PDF) for the two coherent integrators in~\eqref{eq:single_lag_estimator_w_partial_corr}.
By plugging $r_n=s_n\xi e^{j2\pi \delta n}+w_n$ into~\eqref{eq:coherent_integrator}, each of
the two coherent integrators in~\eqref{eq:single_lag_estimator_w_partial_corr} yields a complex Gaussian
random variable (r.v.) with PDF

\begin{equation}
  \label{eq:pdf_co_integrator}
  \begin{aligned}
    &\digamma_l^* \sim \cn\bigg(\frac{E_s\xi^*}{MA^2}\sum_{n=lv}^{(l+1)v-1}e^{-j2\pi \delta n},v\frac{N_0E_s}{2MA^2}\bigg), \\
    &\digamma_{l-k_v} \sim \cn\bigg(\frac{E_s\xi}{MA^2}\sum_{n=(l-k_v)v}^{(l-k_v+1)v-1}e^{j2\pi \delta n},v\frac{N_0E_s}{2MA^2}\bigg).
  \end{aligned}
\end{equation}
where $A^2|s_n|^2 \approx E_s/M$ denotes the average energy per sample.
Note, $\digamma_l^*$ and $\digamma_{l-k_v}$ are uncorrelated. Based on~\eqref{eq:pdf_co_integrator}, 
it is easy to get the product $C_{\digamma_{l}}=\digamma_l^*\digamma_{l-k_v}$ has a mixed distribution with complex Gaussian and a
second kind Bessel function, where the mean $\mu_{C_{\digamma_{l}}}$ and variance $\sigma^2_{C_{\digamma_{l}}}$ are given by, respectively

\begin{equation}
  \begin{aligned}
  \label{eq:mean_var_product_coherent_int}
  \mu_{C_{\digamma_l}}&=\frac{E_s^2}{M^2A^2}\sum_{n=lv}^{(l+1)v-1}e^{-j2\pi \delta n}\bigg(\sum_{m=(l-k_v)v}^{(l-k_v+1)v-1}e^{j2\pi \delta m}\bigg) \\
  &=\frac{E_s^2}{M^2A^2}e^{-j2\pi \delta k_vv}\bigg(\frac{\sin(\pi \delta v)}{\sin(\pi \delta)}\bigg)^2, \\
  \sigma^2_{C_{\digamma_l}}&={\underbrace{v^2\frac{N_0^2E_s^2}{4M^2A^4}}_{\text{from Bessel}}}+{\underbrace{2v\frac{N_0E_s^3}{2M^3A^4}\bigg(\frac{\sin(\pi \delta v)}{\sin(\pi \delta)}\bigg)^2}_{\text{from Complex Gaussian}}},
  \end{aligned}
\end{equation}
where $\frac{\sin(\pi \delta v)}{\sin(\pi \delta)}$ is one of Dirichlet function of $\delta$, which appro-aches
maximum value $v$ at $\delta{=}0$ and first two zeros at $\delta{=}\pm 1/v$.
Note, both $\mu_{C_{\digamma_l}}$, $\sigma^2_{C_{\digamma_l}}$ are independent of $l$.
Thus, $\sum C_{\digamma_l}$ in the argument operator of~\eqref{eq:single_lag_estimator_w_partial_corr} 
also has the mixed distribution with complex Gaussian and a second kind Bessel function.

Recall from~\eqref{eq:single_lag_estimator_w_partial_corr}, the distribution of SL estimator depends on $\arg\{\cdot\}$.
It is derived that the full pdf of $\arg\{\zeta\}$, where $\zeta$ is complex Gaussian distributed, has a good approximation, valid for moderate SNR, as Gaussian.
Specifically,

\begin{equation}
  \label{eq:pdf_arg_comp_Gaus}
  \arg\{\zeta\} \sim \n\big(\angle \mu_{\zeta},\sigma^2_{\zeta}/|\mu_{\zeta}|^2\big).
\end{equation}
The derivation of~\eqref{eq:pdf_arg_comp_Gaus} is omitted due to space constraint.
Now we can try to evaluate the performance of SL estimator based on the result of~\eqref{eq:pdf_arg_comp_Gaus}.
By central limit theorem and assuming a large $N$, the r.v. followed a distribution of second kind Bessel function
is assumed to be complex Gaussian. Thus, an altern-ative performance evaluation 
based on the ratio of (absolute) square of mean to variance of $\sum C_{\digamma_l}$ is given by 

\begin{equation}
  \begin{aligned}
    \label{eq:SNR_out}
    \text{SNR}_{\sum C_{\digamma_l}}=\frac{|\mu_{\sum C_{\digamma_l}}|^2}{\sigma^2_{\sum C_{\digamma_l}}} 
    % &=\frac{\frac{Es^4}{M^4A^4}\Big(\frac{\sin(\pi \delta v)}{\sin(\pi \delta)}\Big)^4(N/v-k_v)^2}
    % {\Big[v^2\frac{N_0^2E_s^2}{4M^2A^4}+2v\frac{N_0E_s^3}{2M^3A^4}\Big(\frac{\sin(\pi \delta v)}{\sin(\pi \delta)}\Big)^2\Big](N/v{-}k_v)} \\
    =\frac{(N/v-k_v)\Big(\frac{\sin(\pi \delta v)}{\sin(\pi \delta)}\Big)^4}
    {\frac{v^2}{\text{SNR}_{in}^2}+\frac{2v}{\text{SNR}_{in}}\Big(\frac{\sin(\pi \delta v)}{\sin(\pi \delta)}\Big)^2}. \\
  \end{aligned}
\end{equation}
where $\text{SNR}_{in}=\frac{2E_s}{MN_0}$. From~\eqref{eq:SNR_out},
we first see at low SNRs, the variance of the second Bessel r.v. kills the $\text{SNR}_{\sum C_{\digamma_l}}$. More-over,  
compared with no correlating, the length-$v$ partial correlating provides an
extra processing gain equaling to $\frac{v+2v\text{SNR}_{in}}{1+2v\text{SNR}_{in}}$ to improve
the performance of the SL estimator; Although, the above processing gain obtained
directly from~\eqref{eq:SNR_out} is very rough, it basically says the 
processing gain is affected by the input SNR and increased by $v$.
While at high SNRs, both est-imators with or without correlating
have the same processing gain from the number of correlations in~\eqref{eq:single_lag_estimator_w_partial_corr}, 
i.e., $N{-}k$, when $|\delta|v \ll 1$. When $|\delta|v$ is large, the SL estimator with partial correlating will have a visible 
degradation of accuracy. Thus, the selection of $v$ trades off the processing gain at low SNRs
and the incurred large frequency offset. 

Furthermore, based on~\eqref{eq:single_lag_estimator_wout_partial_corr},~\eqref{eq:mean_var_product_coherent_int}
and~\eqref{eq:pdf_arg_comp_Gaus}, the distribution of $\hat{\delta}_{SL}^{(v)}(k_v)$ at moderate SNRs is finally given by

\begin{equation}
  \label{eq:lower_bound_single_lag_high_snr}
  \hat{\delta}_{SL}^{(v)}(k_v) \sim \n \Bigg(\delta,\frac{Mv}{4\pi^2k^2(N/v{-}k_v)E_s/N_0\Big(\frac{\sin(\pi \delta v)}{\sin(\pi \delta)}\Big)^2}\Bigg).
\end{equation}
Thus, $\hat{\delta}_{SL}^{(v)}(k_v)$ is unbiased. 
A lower bound for the variance of single-lag estimator at high SNRs is obtained
as the variance of~\eqref{eq:lower_bound_single_lag_high_snr} when $v=1$.
Moreover, we can see the variance of~\eqref{eq:lower_bound_single_lag_high_snr} depends on
the value of $k_v$. The best choice for $k_v$ is to choose 
$k_v=\lfloor\frac{2N}{3v}\rfloor$ to minimize the variance.

\subsubsection{Estimation range}
The SL estimator may suffer the eff-ect
of "aliasing" if $2\pi |\delta|k_vv{>}\pi$. Thus, a safe estimation range for the estimator with optimal $k_v=\lfloor\frac{2N}{3v}\rfloor$ to avoid the modulo-$2\pi$ operation
is $\delta$ within $\pm 3/(4MN)$. Compared with the same autocorrelation-based estimator, e.g.,
the L\&R~\cite{Luise_Reggiannini_95} and Fitz~\cite{Fitz_94}, single-lag estimator has 
$3/8$ estimation range of L\&R and $3/4$ estimation range of Fitz.

\subsubsection{Computational complexity}

We have discussed the accuracy of single-lag estimator.
It is also necessary to compare with the complexity since single-lag estimator
is used in high sample-rate case. The computational complexity of single-lag estimator 
can be readily assessed from~\eqref{eq:single_lag_estimator_w_partial_corr},~\eqref{eq:coherent_integrator} and~\eqref{eq:single_lag_estimator_wout_partial_corr}.

Specifically, we compare with the complexity of SL estimators in single and sequential detection with and without partial correlating.
Note, with partial correlating, the SL estimator has same complexity in single and sequential detection since the
coherent integrator requires the product of the received samples and the preamble 
at the same sample instants. However, with no partial correlating, from~\eqref{eq:single_lag_estimator_wout_partial_corr},
we see $s_{m-k}^*s_m$ can be precomputed and stored
like "filter coefficients"; Moreover, due to the characteristic of the sequential detection process, the products of received samples, $r_{m-k}r_m^*$,
can be stored in a shift register so that only one new product needs to be computed per sample period.

The exact computational complexity of two single-lag estimators are given in Table~\ref{table:computational complexity comparison}. 
We see $\hat{\delta}_{SL}^{(1)}(k)$ has 2 times lower complex products and additions compared with $\hat{\delta}_{SL}^{(v>1)}(k_v)$ in sequential detection.
Furthermore, note the complexity of Kay estimator in~\cite{Morelli_Mengali_98} is given approximately $\frac{3N}{4}$ complex products and additions. By plugging
the optimal $k=\frac{2N}{3}$, $\hat{\delta}_{SL}^{(1)}(k)$ finally has a lower complexity than Kay estimator with $\frac{2N}{3}$ complex products and additions.

\begin{table}[t]
  \caption{Complexity of single-lag estimators with and without partial correlating (complex products~|~additions)}  % title of Table
  \centering 
  \begin{tabular}{c c c} 
  \hline\hline 
   & Single Detection & Sequential Detection \\ [0.5ex] 
  \hline 
  $\hat{\delta}_{SL}^{(1)}(k)$  & 2(N-k)~|~N-k & N-k+1~|~N-k \\ 
  $\hat{\delta}_{SL}^{(v)}(k_v)$ & (2+$\frac{1}{v}$)(N-k)~|~(2-$\frac{1}{v}$)(N-k)-1 & (2+$\frac{1}{v}$)(N-k)~|~(2-$\frac{1}{v}$)(N-k)-1 \\ [1ex]
  \hline
  \end{tabular}
  \label{table:computational complexity comparison}
\end{table}

\subsection{Fine Solution: Newton-Method based Estimator}

The principle of Newton-Method based estimator is to use the single-lag estimator as the starting point for a Newton-type iteration 
aimed at finding a better solution to the necessary condition~\eqref{eq:necessary condition for delta}. 
In principle, multiple iterations are possible to produce successively better approximations to the root of
$J'(\cdot)$ in~\eqref{eq:necessary condition for delta}. Specifically, the iterations are given by

\begin{equation}
  \label{eq:iter_NM_est}
  \hat{\delta}_{NM}^{(i+1)}=\hat{\delta}_{NM}^{(i)}-
  \frac{J(\hat{\delta}_{NM}^{(i)})}{J^\prime(\hat{\delta}_{NM}^{(i)})}
\end{equation}
where $\hat{\delta}_{NM}^{(0)}=\hat{\delta}^{(v)}_{SL}(k_v)$ is the starting point of iteration and
$J^\prime(\cdot)$ denotes the derivative of $J$ with respect to $\hat{\delta}$. 
Our simu-lations indicate that only a single iteration is usually sufficient to achieve very good accuracy.

From~\eqref{eq:iter_NM_est} and the previous discussion, we can conclude the importance of accuracy of single-lag estimator
at low SNRs: 
with a merely sufficient good accuracy, the single-lag estimator not only increases the probability of detection by better
fitting the preamble and received signal as in sequential detector~\eqref{eq:generalized_corr}, but 
it provides a reasonable starting point for getting the more accurate NM estimator. In simluations,
we will also show the case when the NM estimator has a worse accuracy than single-lag estimator if the latter does not provide enough accuracy. 

% ambiguity problem for sl.