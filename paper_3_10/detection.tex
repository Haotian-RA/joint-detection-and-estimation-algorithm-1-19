\section{Detection and Time Synchronization}
\label{sec:detection}

% In this section, we analyze the sequential detection problem, where 
% the detection algorithm proceeds sequentially and each step a window of
% $N$ received samples with $N{-}1$ overlapped samples from the previous window is considered.
We start by looking at the two hypotheses for the sequential detection task:
Let $H_0$ be the null hypothesis that the received signal is the channel noise or only contains a portion of the preamble against the alternative $H_1$ that it contains the entire preamble. 
% The detection algorithm proceeds sequentially and each step a window of
% length $N$ received samples with $N-1$ overlapped samples from the previous window is considered.
Define $\Delta$ to be the distance between the current start position of the observation of $N$ received samples and the start position of the preamble ($p$). 
Thus, for $n=0,1,\ldots,N-1$, the two hypotheses are given by

\begin{equation}
    \label{eq:two hypotheses}
    \begin{aligned}
    &H_0:~r_n = 
    \begin{dcases}
        s_{n-\Delta}Ae^{j\phi}e^{j2\pi\delta (n-\Delta)}+w_n & \text{for}~\Delta\in(0,N-1], \\
        w_n & \text{otherwise},
    \end{dcases} \\
    &H_1:~r_n=s_nAe^{j\phi}e^{j2\pi\delta n}+w_n.
    \end{aligned}
  \end{equation}
% Figure 3 demonstrates the case when $H_0$ represents a portion of the preamble compared with
% $H_1$. 
Note, in~\cite{Ramakrishnan_10,Chiani_06,Liang_15},
the authors define hypothesis $H_0$ only when no preamble is observed.
Although, in~\cite{Chiani_06}, the author points out the effect of partial preamble
can be kindly avoided by us-ing the preamble symbol sequence with a good autocorrelation property.
In this paper, we still focus on discussing when pa-rtial preamble is observed because it is the worst case of $H_0$
and it affects the result of hypothesis testing.

Based on the above discussion, we now build the conditional likelihood ratio test (CLRT) 
between $H_0$ under $\Delta \in(0,N-1]$ and $H_1$ by giving the distance $\Delta$, the phasor $\xi=Ae^{j\phi}$ and
the frequency offset $\delta$ at the position of the preamble (we omit discussing CLRT for $H_0:r_n=w_n$, since it 
is a traditional and well-known hypothesis testing problem). The hypothesis testing is given by

\begin{equation}
    \label{eq:likelihood ratio}
    \begin{aligned}
    \Lambda(R|\Delta,\xi,\delta)&=\frac{f_{R|H_1,\xi,\delta}(r|H_1,\xi,\delta)}{f_{R|H_0,\Delta,\xi,\delta}(r|H_1,\Delta,\xi,\delta)} \\
    &=\frac{\displaystyle \prod_{n=0}^{N-1}\frac{1}{\sqrt{\pi N_0}}e^{-\frac{|r_n-s_n\xi e^{j2\pi\delta n}|^2}{N_0}}}{\displaystyle \frac{1}{(\pi N_0)^{N/2}}\prod_{n=\Delta}^{N-1}e^{-\frac{|r_n-s_{n-\Delta}\xi e^{j2\pi\delta (n-\Delta)}|^2}{N_0}}\prod_{n=0}^{\Delta-1}e^{-\frac{|r_n|^2}{N_0}}} \\
    % &=\frac{\prod_{n=0}^{N-1}\frac{1}{\sqrt{\pi N_0}}e^{-\frac{|r_n-s_n\xi e^{j2\pi\delta n}|^2}{N_0}}}{\prod_{n=\Delta}^{N-1}\frac{1}{\sqrt{\pi N_0}}e^{-\frac{|r_n-s_{n-\Delta}\xi e^{j2\pi\delta (n-\Delta)}|^2}{N_0}}\prod_{n=0}^{\Delta-1}\frac{1}{\sqrt{\pi N_0}}e^{-\frac{|r_n|^2}{N_0}}} \\
    % &\times \frac{1}{\prod_{n=0}^{\Delta-1}\frac{1}{\sqrt{\pi N_0}}e^{-\frac{|r_n|^2}{N_0}}}
    & \LRT{H_1}{H_0} \eta.
    \end{aligned}
  \end{equation}
Cancelling the common parts and taking the logarithm,~\eqref{eq:likelihood ratio} is reduced to

\begin{equation}
    \label{eq:log likelihood}
    \begin{aligned}
    \Re\Bigg\{\sum_{n=0}^{N-1}r_ns_n^*\xi^*e^{-j2\pi\delta n}-\sum_{n=\Delta}^{N-1}r_n&s_{n-\Delta}^*\xi^*e^{-j2\pi\delta(n-\Delta)}\Bigg\} \LRT{H_1}{H_0} \\
    &\frac{N_0}{2}\ln\eta+\frac{A^2}{2}\sum_{n=N-\Delta}^{N-1}|s_n|^2.
    \end{aligned}
\end{equation}
Note, on the left hand side, the two summations perform like the matched filter for hypothesis $H_1$
and $H_0$, respectively. To explain this, we plug $r_n=s_n\xi e^{j2\pi\delta n}+w_n$ under hypothesis $H_1$ into the left hand side of~\eqref{eq:log likelihood}.
In the real operator, it yields

\begin{equation}
    \label{eq:log likelihood under H_1}
    \begin{aligned}
    % &\sigma(\Delta) \\    
    % &=A^2\sum_{n=0}^{N-1}|s_n|^2-A^2e^{j2\pi\delta\Delta}\sum_{n=\Delta}^{N-1}s_ns_{n-\Delta}^*+\text{a zero-mean AWGN}
    A^2\sum_{n=0}^{N-1}|s_n|^2-A^2e^{j2\pi\delta\Delta}\sum_{n=\Delta}^{N-1}s_ns_{n-\Delta}^*+\text{zero-mean noise},
    \end{aligned}
\end{equation}
where $A^2\sum_{n=0}^{N-1}|s_n|^2$ denotes the energy of the preamble and
$\sum_{n=\Delta}^{N-1}s_ns_{n-\Delta}^*$ is the "partial" autocorrelation function (ACF) of the
preamble at lag $\Delta$. Similarly, it can be derived the left hand side of~\eqref{eq:log likelihood}
under hypothesis $H_0$ in the real operator is 
% $-A^2\sum_{n=\Delta}^{N-1}|s_{n-\Delta}|^2+A^2e^{-j2\pi\delta\Delta}\sum_{n=\Delta}^{N-1}s_{n-\Delta}s_n^*$

\begin{equation}
    \label{eq:log likelihood under H_0}
    \begin{aligned}
    % &\sigma(\Delta) \\    
    % &=A^2\sum_{n=0}^{N-1}|s_n|^2-A^2e^{j2\pi\delta\Delta}\sum_{n=\Delta}^{N-1}s_ns_{n-\Delta}^*+\text{a zero-mean AWGN}
    -A^2\sum_{n=\Delta}^{N-1}|s_{n-\Delta}|^2+A^2e^{-j2\pi\delta\Delta}\sum_{n=\Delta}^{N-1}s_{n-\Delta}s_n^*+\text{zero-mean noise}.
    \end{aligned}
\end{equation}
Thus, based on~\eqref{eq:log likelihood under H_1} and~\eqref{eq:log likelihood under H_0},
we can conclude that the principle value of the log-CLRT reduces to the difference between
the energy of the (partial) preamble and the ACF of the preamble sequence at lag $\Delta$. 
By using the symbol sequence $\{c_i\}$ of $s_n$ with a good autocorrelation property, e.g., Gold sequence, the
effect of the ACF can be kindly mitigated; However, because of pulse shaping,
the autocorrelation property of the preamble is decreased so that 
the effect of the ACF cannot be ignored. 

In practice, the distance $\Delta$ is an unknown information to the receiver,
which means the second summation of~\eqref{eq:log likelihood} is not computable.
However, the first summation is computable and it reflects the energy of the preamble under
$H_1$ and the "partial" ACF under $H_0$.
Thus, a practical sequential detector is built just based on the classical 
cross-correlation between the received signal
and the preamble corrected by the frequency and phasor estimates with some proper scaling. Specifically,

\begin{equation}
    \label{eq:generalized_corr}
    \rho(\tilde{p})=
    \frac{\Re\{\langle
      \bm{r}_{\tilde{p}},\hat{\bm{s}}_{\tilde{p}}\rangle\}}
    {||\bm{r}_{\tilde{p}}||\cdot||\hat{\bm{s}}_{\tilde{p}}||} \LRT{H_1}{H_0} \gamma
  \end{equation}
where $\tilde{p}{=}p{+}\Delta$ denotes the general start position of received signal. 
$\bm{r}_{\tilde{p}}{=}[r_{\tilde{p}},r_{\tilde{p}+1},\ldots,r_{\tilde{p}+N-1}]$ represents the received signal sequence at
the position $\tilde{p}$, and $\hat{\bm{s}}_{\tilde{p}}$ represents the carrier-estimates corrected preamble sequence, where each element is 
$\hat{s}_{n}=s_n\hat{\xi}_{\tilde{p}}e^{j2\pi\hat{\delta}_{\tilde{p}}n}~\text{for}~n=0,1,\ldots,N-1$, and $\hat{\xi}_{\tilde{p}}$, $\hat{\delta}_{\tilde{p}}$
are the phasor and frequency estimates at position $\tilde{p}$. Moreover, $||\cdot||$ is the Euclidean norm of the signal sequence,
and $\gamma$ is the normalized detection threshold which lies on the range of $[0,1]$. 
From~\eqref{eq:generalized_corr}, it is obvious that to implement the sequential detector, we need the information
of the frequency and phasor estimates at that position. Thus, a realistic generalized likelihood ratio test (GLRT) replaces the CLRT of~\eqref{eq:likelihood ratio}
by first doing carrier synchronization at each position of detection then plugging the estimates into the LRT.
Furthermore, it should be emphasized that our sequential detector requires to work at high sample rate; Although~\eqref{eq:generalized_corr}
is already the simpliest form, the complexity of the two estimates $\hat{\xi}$,$\hat{\delta}$ is also crucial. 
A pair of low-complexity frequency and phasor estimates will be given in the next section.





