\section{Detection and Time Synchronization}
\label{sec:detection}

As illustrated in Figure~\ref{fig:sig_acquis_chain}, the first step to analyze
the signal acquisition chain is to formulate the detector.
The algorithm proceeds sequentially and each step a window of length $N$ (the number of samples in the preamble) received samples with 
$N-1$ overlapped samples is considered. The sequential detection problem solved 
in this paper is fundamentally equivalent to sequential frame synchronization~\cite{Massey_72,Lui_Tan_86,Scholtz_80}.

The detection algorithm is based on a Generalized Likelihood Ratio Test (GLRT): 
Let $H_0$ be the null hypothesis that the preamble is not completely in the obeservation window against the alternative 
$H_1$ that it does. We further define $\Delta$ to be the distance between the current sample instant $p_c$ and the true position of preamble $p$, 
i.e., $\Delta=p-p_c$. Thus, we can translate $H_0$ as when $\Delta \neq 0$ while $H_1$ as when $\Delta=0$.
The likelihood ratio test between $H_0$ and $H_1$ is given by

\begin{equation}
    \label{eq:likelihood ratio}
    \begin{aligned}
    \Lambda(R)=\frac{p_{R|H_1}(r|H_1)}{p_{R|H_0}(r|H_0)}&= 
    \frac{\displaystyle \prod_{n=0}^{N-1}\frac{1}{\sqrt{\pi N_0}}\exp(-\frac{1}{2}\frac{|r_n-s_nSb^n|^2}{N_0/2})}
    {\displaystyle \prod_{n=\Delta}^{N-1}\frac{1}{\sqrt{\pi N_0}}\exp(-\frac{1}{2}\frac{|r_n{-}s_{n-\Delta}Sb^n|^2}{N_0/2})} \cdot\\
    &\frac{1}{\displaystyle \prod_{n=0}^{\Delta-1}\frac{1}{\sqrt{\pi N_0}}\exp(-\frac{1}{2}\frac{|r_n|^2}{N_0/2})}
    \LRT{H_1}{H_0} \eta,
    \end{aligned}
\end{equation}
where we define $S=Ae^{j\phi}$, $b=e^{j2\pi \delta}$ for notation simplicity.
Cancelling the common parts and taking the logarithm,~\eqref{eq:likelihood ratio} is reduced to

\begin{equation}
    \label{eq:log likelihood}
    \begin{aligned}
    \Re\{\sum_{n=0}^{N-1}r_ns^*_nS^*b^{-n}-
    \sum_{n=\Delta}^{N-1}r_ns^*_{n-\Delta}S^*b^{-n}\}
    \LRT{H_1}{H_0} \frac{N_0}{2}&\ln\eta+ \\
    &\frac{A^2}{2}\sum_{n=N-\Delta}^{N-1}|s_n|^2.
    \end{aligned}
\end{equation}
The minuend of left hand side of~\eqref{eq:log likelihood} points to the inner product of the overlap between
the observed (partial) preamble and the true preamble. 
By plugging $r_n$ of hypothesis $H_1$, i.e., $r_n=s_nSb^n+w_n$, the minuend simplified as

\begin{equation}
    \label{eq:partial ACF of the preamble}
    \sigma(\Delta)=A^2\sum_{n=\Delta}^{N-1}s_ns^*_{n-\Delta}+\sum_{n=\Delta}^{N-1}w_ns^*_{n-\Delta}S^*b^{-n}
\end{equation}
is a Gaussian random variable with mean as a "partial" auto-correlation function (ACF) of the preamble at lag $\Delta$.
In order to quantify~\eqref{eq:partial ACF of the preamble}, the symbol sequence of the preamble should be chosen with a good autocorrelation property,
such as Gold sequence, that the partial ACF is approximately equal to zero for all $\Delta \neq 0$. 
Thus, after some proper scaling of~\eqref{eq:log likelihood}, the GLRT finally reduces to 
a function with respect to the current time instant $p$,

\begin{equation}
    \label{eq:generalized_corr}
    \frac{\Re\{\langle
      \bm{r}_p,\hat{\bm{s}}\rangle\}}
    {||\bm{r}_p||\cdot||\hat{\bm{s}}||} \LRT{H_1}{H_0} \gamma
  \end{equation}
where $\hat{\bm{s}}$ denote the carrier-estimate corrected reference samples, i.e., $\hat{s}_{n}~{=}~s_{n}Ae^{j(2\pi\hat{\delta}n+\hat{\phi})}$ for $n=0,\ldots,N{-}1$.
$r_p$ are the current observed received samples.
Thus, we can intepret~\eqref{eq:generalized_corr} as the generalized correlation between current observed burst and carrier-estimates corrected reference sequence.
$\gamma$ is the detection threshold which lies on the range of $[0,1]$.

We see that building the GLRT detector of~\eqref{eq:generalized_corr} relies on the frequency and phase estimates (carrier synchronization) first.
Furthermore, since the sequential detection proceeds at every time instant, to make it work in practice, the complexity of the carrier estimates becomes
much crucial. A low computational-complexity estimator should be derived for detection purpose.

%
% \subsection{Fractional Delay}
% don't know we should talk about the fractional delay or not for the time being. Just ignore it for a second.
% 