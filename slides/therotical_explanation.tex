\section{Joint Synchronization Algorithm}

\subsection{Signal Model}

\begin{frame}
  \frametitle{Signal Model}
  \begin{itemize}
    % \item In our model, we assume a training sequence (preamble) is known to the receiver (DA operation). 
    % \item The goal is to estimate accurately the start time of the preamble and to estimate carrier phase and frequency offsets from the preamble. 
    \item The received samples $r_n$ are given by
    \begin{equation*}
      \begin{aligned}
        \label{eq:model}
        r_n = s_{n-\bar{p}}Ae^{j\phi}e^{j2\pi\delta n}+w_{n},
      \end{aligned}
    \end{equation*}
    where
    \begin{itemize}
        % \item $p$ is the global time axis measuring the position of samples in the entire received stream.
        \item $s_n$: the pulse-shaped preamble for $n=0,1,\ldots,N{-}1$, where $N$ is the number of samples.
        \item $\bar{p}$: the start position of the received training sequence.
        \item $\xi=Ae^{j\phi}$: the unknown phasor offset.
        \item $\delta$: the unknown frequency offset.
        \item $w_n$: complex addition white Gaussian noise (AWGN).
    \end{itemize}
    \item The training sequence (preamble) is known to the receiver (DA operation). 
    \item The goal is to accurately detect the preamble in received data sequence and estimate $\xi$ and $\delta$ from the preamble.
\end{itemize}
  
\end{frame}

\subsection{Time Synchronization}

\begin{frame}
    \frametitle{Time Synchronization}

    \begin{center}
      \includegraphics[width=.6\textwidth]{partial_preamble_detection_in_one_plot.png}
    \end{center}

    \begin{itemize}
      \item Two hypotheses for sequential detection task are given by:
      % using sliding window to explain
      
      \begin{equation*}
        \label{eq:two hypotheses}
        \begin{aligned}
        &H_0{:}~r_n{=}
        \begin{dcases}
            s_{n-\Delta}\xi e^{j2\pi\delta (n-\Delta)}{+}w_n & \max(0{,}\Delta){\leq} n {<} \min(N{,}N{+}\Delta) \\
            w_n & \text{otherwise},
        \end{dcases} \\
        &H_1{:}~r_n{=}s_n\xi e^{j2\pi\delta n}+w_n, \quad n = 0,1,\ldots,N{-}1,
        \end{aligned}
      \end{equation*}
      where
      
      \begin{itemize}
          % \item $H_0$: the received signal is the channel noise or only contains a portion of the preamble.
          % \item $H_1$: the received signal contains the entire preamble.
          \item $\Delta$ is the distance between the current start position of observation window and the true position of the preamble.
      \end{itemize}

    \end{itemize} 

\end{frame}


\begin{frame}
    \frametitle{Hypothesis Testing}
    \begin{itemize}

      \item The conditional likelihood ratio test (CLRT) is conditioned on $\Delta$, the phasor $\xi$, and the frequency offset $\delta$
  
      \begin{equation*}
          \label{eq:log likelihood}
          \begin{aligned}
          \Re\Bigg\{\sum_{n=0}^{N-1}r_ns_n^*\xi^*e^{-j2\pi\delta n}-\sum_{n=\Delta}^{N-1}r_n&s_{n-\Delta}^*\xi^*e^{-j2\pi\delta(n-\Delta)}\Bigg\} \LRT{H_1}{H_0} \\
          &\frac{N_0}{2}\ln\eta+\frac{A^2}{2}\sum_{n=N{-}\Delta}^{N-1}|s_n|^2.
          \end{aligned}
      \end{equation*}
  
      \begin{itemize}
          \item The two inner products on left hand side are the
            matched filters for hypothesis $H_1$ and $H_0$,
            respectively. 
          \item The matched filter for $H_0$ is not computable in
            practice due to unknown time offset  $\Delta$.
          \item The matched filter for $H_0$ must be rendered
            negligible by using a preamble with very good
            autocorrelation properties. 
      \end{itemize}
      
  \end{itemize}

\end{frame}

\begin{frame}
  \frametitle{The Sequential Detector: GLRT}
    \begin{itemize}
    
        \item A practical sequential detector is then derived for each observation window starting at sample $p$
        
        \begin{equation*}
          \label{eq:generalized_corr}
          \rho(p)=
          \frac{\Re\{\langle
            \boldsymbol{r}_{p},\hat{\boldsymbol{s}}_{p}\rangle\}}
          {||\boldsymbol{r}_{p}||\cdot||\hat{\boldsymbol{s}}_{p}||} \LRT{H_1}{H_0} \gamma
        \end{equation*}

        where 
        \begin{itemize}
          \item $\bm{r}_{p}=[r_{p},r_{p+1},\ldots,r_{p{+}N{-}1}]$ denotes the received samples in observation window
          at sample $p$.
          \item $\hat{\bm{s}}_{p}$ denotes the carrier-corrected preamble,
          where each element is $\hat{s}_{n}=s_n\hat{\xi}_{p}e^{j2\pi\hat{\delta}_{p}n}$
          for $n=0,1,\ldots,N{-}1$.
          \begin{itemize}
            \item $\hat{\xi}_{p}$ is the phasor estimate, and 
            \item $\hat{\delta}_{p}$ is the frequency estimate at sample $p$.
          \end{itemize}          

        \end{itemize}

        \item The complexity of estimating $\hat{\xi}$, $\hat{\delta}$ is critical for practical implementation.         
        
    \end{itemize}

\end{frame}

\subsection{Carrier Synchronization}

\begin{frame}
  \frametitle{Carrier Synchronization}
    \begin{itemize}
    
       \item Assume that the time synchronization is perfect.
       \item The maximum likelihood estimation (MLE) of the unknown parameters in the signal model is given by
       
       \begin{equation*}
        \label{eq:ML_f_xi}
        \hat{\delta},\hat{\xi}=\argmin_{\delta,\xi=Ae^{j\phi}}\sum_{n=0}^{N-1}\Big|r_n-s_n\xi e^{j2\pi\delta n}\Big|^{2}.
      \end{equation*}

      \item The closed form for $\hat{\xi}$ and a necessary condition for $\hat{\delta}$ are given by
      
      % and a necessary condition for $\hat{\delta}$ are obtained by taking the Wirtinger derivative
      
      \begin{equation*}
        \label{eq:opt_xi}
        \hat{\xi}=\frac{\sum_{n=0}^{N-1}{r_{n}s_n^{*}e^{-j2\pi\hat{\delta} n}}}{\sum_{n=0}^{N-1}|s_{n}|^2},
      \end{equation*}

      \begin{equation*}
        \label{eq:necessary condition for delta}
        J(\hat{\delta}) = \Im\bigg\{\sum_{k=1}^{N-1}{\sum_{m=k}^{N-1}{kr_{m-k}r_m^{*}s_{m-k}^{*}s_m}e^{j2\pi\hat{\delta}k}}\bigg\}=0.
      \end{equation*}

    \end{itemize}


\end{frame}


\begin{frame}
  \frametitle{Single-lag Estimator with Partial Coherent Integration}

    \begin{itemize}
    
      \item At moderate SNR, any lag $k$ can be used to estimate the frequency offset $\delta$. 
      
      % \begin{itemize}
      %   \item This suggests that an unbiased estimate of $\delta$ can be obtained by using only a single lag $k$.
      % \end{itemize}
      
      \item To extend the operation to low SNR, a generalized single-lag (SL) estimator with partial coherent integration (PCI) is derived
      
      \begin{equation}
        \label{eq:single_lag_estimator_w_partial_corr}
        \hat{\delta}_{SL}^{(v)}(k_v)=-\frac{\arg\big\{\sum_{l=k_v}^{N/v-1}\digamma_l^{*(v)}\digamma_{l-k_v}^{(v)}\big\}}{2\pi k_vv},
      \end{equation}

      where $\digamma_{l}^{(v)}$ is coherent integration over block $l$
      of length $v$

      \begin{equation}
        \label{eq:coherent_integrator}
        \digamma_l^{(v)}=\frac{1}{v}\sum_{n=lv}^{(l+1)v-1}r_ns_n^*, \quad \text{for}~l=0,1,\ldots,N/v{-}1.
      \end{equation}

      and 
      $k_v=\lfloor k/v \rfloor$.
     
    \end{itemize}

    % \begin{center}
    %   \includegraphics[width=.5\textwidth]{general_SD_estimator.png}
    % \end{center}


\end{frame}



\begin{frame}
  \frametitle{Block Diagram}

    \begin{center}
      \includegraphics[width=.9\textwidth]{general_SD_estimator_slides.png}
    \end{center}

    % \begin{itemize}
    
    %   \item The above figure illustrates how Single-Lag estimator with PCI works.
    %   \item The intention of PCI is to increase the SNR of coherent terms before estimating frequency offset from non-coherent terms by leveraging the processing gain.
    %   \item The complexity is degraded by $O(N)$.

    % \end{itemize}

    % \begin{center}
    %   \includegraphics[width=.5\textwidth]{general_SD_estimator.png}
    % \end{center}


\end{frame}


\begin{frame}
  \frametitle{Properties of Non-coherent Products}

    \begin{itemize}
    
      \item Define $C_{\digamma}(v,l)=\digamma_l^*\digamma_{l-k_v}$. The cross correlation has a mixed distribution with mean and variance

      \begin{equation*}
        \begin{aligned}
        \label{eq:mean_var_product_coherent_int}
        \mu_{C_{\digamma}}&=\frac{E_s}{Mv^2}\bigg(\sum_{n=lv}^{(l+1)v-1}e^{-j2\pi \delta n}\bigg)\bigg(\sum_{m=(l-k_v)v}^{(l-k_v+1)v-1}e^{j2\pi \delta m}\bigg) \\
        &=E_s/M\cdot e^{-j2\pi \delta k_vv}\cdot\D^2(v,\delta)/v^2, \\
        \sigma^2_{C_{\digamma}}&={\underbrace{N_0^2/(4v^2)}_{\text{from Bessel}}}+{\underbrace{N_0E_s/(Mv)\cdot\D^2(v,\delta)/v^2}_{\text{from Complex Gaussian}}},
        \end{aligned}
      \end{equation*}

      \begin{itemize}
        \item $\D(v,\delta)=\frac{\sin(\pi \delta v)}{\sin(\pi\delta)}$ 
        % is the Dirichlet function of $\delta$, which 
        has the maximum value $v$ at $\delta=0$.
        % \item $E_s/N_0$ is the symbol energy to noise power ratio (SNR).
        % \item The value of $v$ trades off the output SNR.
        \item The value of $v$ trades off the mean and variance of $C_{\digamma}$ 
        \begin{itemize}
          \item $\mu_{C_{\digamma}}$ increases as $v$ decreases, and
          \item $ \sigma^2_{C_{\digamma}}$
            % of Bessel term (noise square) % both variane terms decrease with v!
            decreases as $v$ increases.
        \end{itemize}
      \end{itemize}
    

    \end{itemize}

    % \begin{center}
    %   \includegraphics[width=.5\textwidth]{general_SD_estimator.png}
    % \end{center}


\end{frame}


% \begin{frame}
%   \frametitle{Performance improvement by PCI}

%     \begin{itemize}

%       \item One way to look at the performance of the SL estimator with PCI is by the "output" SNR
      
%       \begin{equation}
%         \begin{aligned}
%           \label{eq:SNR_out}
%           \text{SNR}_{\sum C_{\digamma}}^{(v,\delta)}=\frac{|\mu_{\sum C_{\digamma}}|^2}{\sigma^2_{\sum C_{\digamma}}} 
%           =\frac{(N/v-k_v)\cdot\D^4(v,\delta)}
%           {v^2/\text{SNR}_{\text{in}}+2v\cdot\D^2(v,\delta)}\cdot\text{SNR}_{\text{in}}. \\
%         \end{aligned}
%       \end{equation}
    
%       \begin{itemize}
%         \item The "output" SNR is degraded by the variance of the second kind Bessel random variable.
%       \end{itemize}

%       \item To see PCI improves the performance of SL at low SNR via the ratio
      
%       \begin{equation}
%         \begin{aligned}
%           \label{eq:relative_processing_gain}
%           \frac{\text{SNR}_{\sum C_{\digamma}}^{(v,0)}}{\text{SNR}_{\sum C_{\digamma}}^{(1,0)}}
%           \approx\frac{v+2v\cdot\text{SNR}_{\text{in}}}{1+2v\cdot\text{SNR}_{\text{in}}}.
%         \end{aligned}
%       \end{equation}

%       \begin{itemize}
%         \item the relative performance increases with $v$ at very low SNR.
%       \end{itemize}

%     \end{itemize}

%     % \begin{center}
%     %   \includegraphics[width=.5\textwidth]{general_SD_estimator.png}
%     % \end{center}


% \end{frame}


\begin{frame}
  \frametitle{Newton-Method Based Estimator}

    \begin{itemize}
    
      \item Using the SL estimator as starting point, a Newton-type
        iteration is aimed at finding a better solution to the necessary condition for $\hat{\delta}$      
      \begin{equation*}
        \label{eq:iter_NM_est}
        \hat{\delta}_{NM}^{(i+1)}=\hat{\delta}_{NM}^{(i)}-
        \frac{J(\hat{\delta}_{NM}^{(i)})}{J^\prime(\hat{\delta}_{NM}^{(i)})}
      \end{equation*}
      where
      \begin{itemize}
        \item $\hat{\delta}_{NM}^{(0)}=\hat{\delta}^{(v)}_{SL}(k_v)$.
        \item $J^\prime(\cdot)$ is the derivative of objective
          function $J$ with respect to $\hat{\delta}$.
        \item Complexity is $O(N^2)$.
      \end{itemize}
      \item A single iteration is usually sufficient to achieve very
        good accuracy.
      \item Refinement is computed only when signal is detected.
      % \item Summary: The SL estimator is used for operating at high sample rate for sequential detection task due to its low complexity.
      % The NM estimator improves the accuracy after reliable detection to enable coherent demodualtion.
        

    \end{itemize}




\end{frame}

\subsection{Simulation}

\begin{frame}
  \frametitle{Accuracy of SL and NM Estimators}

    \begin{center}
      \includegraphics[width=.57\textwidth]{accuracy_NM_SL_slides.png}
    \end{center}

    \begin{itemize}
    
      \item The accuracy of SL and NM are improved by using PCI.
      \item The accuracy of SL increases as frequency error decreases;
        due to  larger Dirichlet function $\D(v,\delta)$.  
      \item NM approaches the CRVB when SL estimate is sufficiently accurate.

    \end{itemize}




\end{frame}

\begin{frame}
  \frametitle{Receiver Operating Characteristics}

    \begin{center}
      \includegraphics[width=.6\textwidth]{ROC_new_slides.png}
    \end{center}

    \begin{itemize}
    
      % \item The above figure shows the receiver operating characteristics (ROC) of the sequential detector.
      \item The better accuracy of SL with PCI also improves the detection performance at low SNR.
      \item Good detection performance at low SNR requires a longer preamble.
    \end{itemize}

\end{frame}