\section{Detection and Time Synchronization}
\label{sec:detection}

As illustrated in Figure~\ref{fig:sig_acquis_chain}, the first step to analyze
the signal acquisition chain is to formulate the detector.
The detection algorithm proceeds sequentially and each step a window of length $N$ received samples with 
$N-1$ overlapped samples of the previous window is considered. The sequential detection problem solved 
in this paper is fundamentally equivalent to sequential frame synchronization~\cite{Massey_72,Lui_Tan_86,Scholtz_80}.
Since we mainly focus on the detection algorithm with respect to the timing instant (delay), to make analysis simpler,
the fractional delay of~\eqref{eq:model} is neglected by assuming a sufficient large sample rate in this section. 
The effect of fractional delay on detection will be discussed by looking at the receiver operating characteristics (ROC) in simulation (Section~\ref{sec:simulations}).

The detection algorithm is based on a Generalized Likelihood Ratio Test (GLRT).
Let $H_0$ be the null hypothesis that the preamble is not completely in the obeservation window against the alternative 
$H_1$ that it does. We define $\Delta$ to be the distance between the current position of the received sequence $p_c$ and the true position of preamble $p$, 
i.e., $\Delta=|p-p_c|$. Figure~\ref{fig:} illustrates when the current received sequence in the observation window only shows partial received preamble.
The progress of analyzing the received sequence in the observation window having passed
true position of the preamble ($p_c>p$) is symmtry. Thus, we can translate $H_0$ as when $\Delta \neq 0$ while $H_1$ as when $\Delta=0$.
Based on~\eqref{eq:model}, we first build the conditional likelihood ratio test (LRT) between $H_0$ and $H_1$ given the pha-sor $S{=}Ae^{j\phi}$ and frequency offset $b{=}e^{j2\pi \delta}$
at true delay $p$,

\begin{equation}
    \label{eq:likelihood ratio}
    \begin{aligned}
    &\Lambda(R|S,b)=\frac{p_{R|H_1,S,b}(r|H_1,S,b)}{p_{R|H_0,S,b}(r|H_0,S,b)} \\
    &=\frac{\displaystyle \prod_{n=0}^{N-1}\frac{1}{\sqrt{\pi N_0}}e^{(-\frac{1}{2}\frac{|r_{n+p}{-}s_nSb^n|^2}{N_0/2})}}
    {\displaystyle \prod_{n=\Delta}^{N-1}\frac{1}{\sqrt{\pi N_0}}e^{(-\frac{1}{2}\frac{|r_{n+p-\Delta}{-}s_{n-\Delta}Sb^{n-\Delta}|^2}{N_0/2})} 
    \displaystyle \prod_{n=0}^{\Delta-1}\frac{1}{\sqrt{\pi N_0}}e^{(-\frac{1}{2}\frac{|r_n|^2}{N_0/2})}} \LRT{H_1}{H_0} \eta.
    \end{aligned}
\end{equation}
Cancelling the common parts and taking the logarithm,~\eqref{eq:likelihood ratio} is reduced to

\begin{equation}
    \label{eq:log likelihood}
    \begin{aligned}
    \Re\{\sum_{n=0}^{N-1}r_{n+p}s^*_nS^*b^{-n}-
    \sum_{n=\Delta}^{N-1}r_{n+p-\Delta}s^*_{n-\Delta}S^*&b^{-(n-\Delta)}\}
    \LRT{H_1}{H_0} \frac{N_0}{2}\ln\eta \\
    &+\frac{A^2}{2}\sum_{n=N-\Delta}^{N-1}|s_n|^2.
    \end{aligned}
\end{equation}
The second summation on the left hand side points to the inner product of the overlap between
the observed (partial) preamble and the true preamble. 
By plugging $r_{n+p-\delta}$ of hypothesis $H_1$, i.e., $r_{n+p-\delta}=r_{n+p}=s_nSb^n+w_n$, the summation simplifies 
to a function $\sigma$ in terms of $\Delta$

\begin{equation}
    \label{eq:partial ACF of the preamble}
    \sigma(\Delta)=A^2b^{\Delta}\sum_{n=\Delta}^{N-1}s_ns^*_{n-\Delta}+\sum_{n=\Delta}^{N-1}w_ns^*_{n-\Delta}S^*b^{-(n-\Delta)},
\end{equation}
which is Gaussian distributed with mean given  "partial" auto-correlation function (ACF) of the preamble at lag $\Delta$.
To quantify~\eqref{eq:partial ACF of the preamble}, the symbol sequence of the preamble can be chosen with a good autocorrelation property,
e.g., Gold sequence, that the partial ACF is approximately equal to zero for all $\Delta \neq 0$. 
% from here, it gives me a new and very important observation: (7) may explain why we obtain the estimator
% by assuming the known position of preamble (p_c=p) is meaningful. (7) and (8) tells us we can
% the detector works when plugping in the estimator by assuming the current position is the true position.
% then (6) -> (8).
By assuming that, after some proper scaling of~\eqref{eq:log likelihood}, the GLRT finally reduces to 
a generalized correlation function in terms of the time instant $p_c$,

\begin{equation}
    \label{eq:generalized_corr}
    \rho(p_c)=
    \frac{\Re\{\langle
      \bm{r}_{p_c},\hat{\bm{s}}_{p_c}\rangle\}}
    {||\bm{r}_{p_c}||\cdot||\hat{\bm{s}}_{p_c}||} \LRT{H_1}{H_0} \gamma
  \end{equation}
where $\hat{\bm{s}}_{p_c}$ denotes the carrier-estimate corrected preamble, i.e., $\hat{s}_{p_c}[n]~{=}~s_{n}\hat{S}_{p_c}\hat{b}_{p_c}^n$ for $n=0,\ldots,N{-}1$,
at delay $p_c$. $\hat{b}_{p_c}$ and $\hat{S}_{p_c}$ are the frequency and phasor estimates with respect to $p_c$.
$\bm{r}_{p_c}$ represents the received data sequence at delay $p_c$.
$\gamma$ is the detection threshold which lies on the range of $[0,1]$.
Note that the LRT in~\eqref{eq:likelihood ratio} is not practical since the frequency and phasor offsets at true delay $p$
is unknown while detection. Instead of LRT, GLRT can be built relying on the frequency and phasor estimates from each window of received sequence;
Recall that the derivation from~\eqref{eq:log likelihood} to~\eqref{eq:partial ACF of the preamble} is based on assuming the partial preamble be the true preamble.
Therefore, the estimating algorithm of $\hat{b}_{p_c}$ and $\hat{S}_{p_c}$ for each window at $p_c$ should also be derived by
assuming the received sequence contains the true preamble.
% this tells the reason why we do estimation by assuming the position of preamble is known.

In conclusion,the GLRT based detector of~\eqref{eq:generalized_corr} relies on the frequency and phase estimates of the observed sequence first.
Furthermore, since the sequential detection proceeds at every time instant, to make it work in practice, the complexity of the carrier estimates becomes
much crucial. A low computational-complexity estimator should be derived for detection purpose.

%
% \subsection{Fractional Delay}
% don't know we should talk about the fractional delay or not for the time being. Just ignore it for a second.
% 