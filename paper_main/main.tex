\documentclass[10pt,final,conference,comsoc]{IEEEtran}

% reviewer 1, comment 1 and 7 haven't been extended.
% Some figure maybe need to be fixed a little bit.
\usepackage[utf8]{inputenc}

\usepackage[T1]{fontenc} % optional
\usepackage{amsmath,amssymb,amsfonts}
\usepackage[cmintegrals]{newtxmath}
\usepackage{cite}
\usepackage{algorithmic}
\usepackage{graphicx}
\usepackage{textcomp}
\usepackage{xcolor}
\usepackage{bm}
\usepackage{siunitx}
\usepackage{comment}
\usepackage{mathtools}
\usepackage{url}

\def\BibTeX{{\rm B\kern-.05em{\sc i\kern-.025em b}\kern-.08em
    T\kern-.1667em\lower.7ex\hbox{E}\kern-.125emX}}

\newcommand{\LRT}[2]{
  \mathrel{\mathop\gtrless\limits^{#1}_{#2}}
}
\newcommand{\numb}[1]{\SI{#1}}
\newcommand{\dB}{\decibel}
\newcommand{\est}[2]{\hat{#1}_{#2}}
\newcommand{\sd}{\text{SD}}
\newcommand{\cn}{\mathcal{CN}}
\newcommand{\n}{\mathcal{N}}
\newcommand{\opt}{\text{opt}}
\newcommand{\nm}{\text{NM}}
\newcommand{\J}{\mathrm{J}}
\newcommand{\prob}[1]{\text{P}_{#1}}
\newcommand{\fa}{\text{FA}}
\newcommand{\D}{\text{D}}
\newcommand{\E}{\mathbb{E}}
\newcommand{\Q}{\mathrm{Q}}
\newcommand{\Lagr}{\mathcal{L}}


\graphicspath{{../main_matlab_figures/}{../matlab/}}

\DeclareUnicodeCharacter{0303}{}

\begin{document}

\title{Low Complexity Methods for Joint Detection and Synchronization of TDMA Bursts
}

\author{\IEEEauthorblockN{Haotian Zhai 
  and Bernd-Peter Paris}
\IEEEauthorblockA{\textit{Department of Electrical and Computer Engineering} \\
\textit{George Mason University}\\
Fairfax, VA 22030 \\
\{hzhai,pparis\}@gmu.edu}
}

\maketitle


\begin{abstract}

This paper proposes a data-aided joint detection and synchronization algorithm for TDMA bursts.
A sequential detection algorithm based on the Generalized Likelihood Ratio Test (GLRT) is used to 
detect the embedded preamble signal in received data stream. 
% sequential detection algorithm is necessary to be derived.
Carrier synchronization is attempted for each sample instant during sequential detection and
resulting phase and frequency estimates are used by the GLRT. 
To make this algorithm implemented in software-defined radio (SDR),
a low complexity carrier estimation algorithm with good estimation accuracy is proposed.
Then, a refinement of the carrier estimate is computed when the preamble has been detected.
% SDR is also mentioned in abstract.
It is shown that this estimate approaches the Cramer-Rao bound even at low SNR. 
The complete joint detection and estimation procedure is validated through simulations and SDR.

\end{abstract}

\begin{IEEEkeywords}
Joint detection and estimation, GLRT, low computational complexity, Cramer-Rao vector bound, Software-defined radio.
\end{IEEEkeywords}

\section{Introduction}%
\label{sec:introduction}

In digital communication systems, information is commonly transmitted in time-multiplexed bursts.
% Examples include time-slotted random access systems.
Each active user transmits information to the receiver
in the same frequency band and in non-overlapping time intervals~\cite{Falconer_95}. 
Our methods are pertinent for all TDMA systems where the arrival time
of the burst and the carrier frequency and phase are
not perfectly known.
Obvious examples include random access channels, including those in cellular
systems. Even when a coarse timing structure is imposed, as in slotted
TDMA systems, the precise arrival time of a signal may not be known;
this is generally true for the uplink of cellular-like systems. 
In such cases, the methods presented herein are still relevant  by
restricting the search for the start of the burst to periods near the
start of a TDMA~slot. 

A fundamental prerequisite for successful coherent demodulation is that the receiver can
detect the beginning of the data stream and estimate accurately the phase and frequency offset of the carrier.
It is worth emphasizing that time and carrier synchronization are \emph{coupled problems}, especially at low SNRs:
coherent methods for detecting the signal require accurate frequency and phase estimates 
while data-aided frequency and phase estimation requires that the location of the training sequence is available.
Thus, joint signal detection and carrier synchronization algorithms play a vital role in any communication system.

Clearly, the signal acquisition problem has been considered widely. 
In the late 90's, Morelli and Mengali~\cite{Morelli_Mengali_98} presented a tutorial review
of the carrier synchronization field comparing such characteristics as estimation accuracy, range,
and computational complexity of available techniques.
The work by~\cite{kay_89,Fitz_94,Luise_Reggiannini_95}
is most closely related to results in this paper.
For signal detection and
particularly, sequential detection, the detector
is commonly derived from hypothesis testing principles~\cite{Ramakrishnan_10,Chiani_06,Liang_15}.
When the detector is lacking parameters to construct 
likelihood ratio test (LRT), generalized likelihood ratio tests (GLRT)
are employed to cope with the lack of information, e.g.,~\cite{Chiani_06}. 

In this paper, we propose a family of joint detection and estimation algorithm 
for the complete signal acquisition process.
We emphasize computational complexity while maintaining near-optimal performance.
The practicality of our algorithms is demonstrated by implementing
them on a standard SDR platform; in our experiments we can sustain
nearly \SI{10}{\mega\hertz} sample rates while continuously searching
for the presence of the preamble signal.


\section{Signal Model}
\label{sec:model}

\begin{figure*}[t]
  \centerline{\includegraphics[width=6.5in]{signal_acquisition_chain.png}}
  \caption{Block diagram for analysis of the complete signal acquistion chain}
  \label{fig:sig_acquis_chain}
  \end{figure*}

The transmitted signal burst is assumed to include a refere\-nce signal that is known to the receiver.
Often such a reference sequence is prepended to the payload and is referred to as a preamble.
The problem addressed in this paper is to accurately estimate the start time of the preamble and 
to estimate carrier phase and frequency offset from this preamble.
Hence, the payload portion of the burst is not further considered.

The baseband-equivalent reference signal $s(t)$ is digitally modulated as
\begin{equation}
    \label{eq:l_ref_sig_analog}
    s(t) = \sum_{i=0}^{L_0-1} c_i g(t-iT),
  \end{equation}
where $T$ denotes the symbol period and $g(t)$ provides pulse shaping. $c_i$ are the known symbol sequence
between transmitter and receiver, and $L_0$ is the number of symbols. It is common to assume the symbols 
$\{c_i\}$ have good autocorrelation properties to render coherent processing effective.

The received baseband equivalent signal stream are given by
\begin{equation}
    \label{eq:rec_sig_analog}
    r(t) = s(t-\tau) \cdot Ae^{j \phi} e^{j2\pi f_d t} + w(t).
  \end{equation}
where $\tau$ denotes the delay (the start time) of received reference signal in stream. $A$, $\phi$, $f_d$ 
are the carrier amplitude, phase, and frequency offset, respectively. These are the parameters to be estimated
for signal acquisition. The complex, additive white Gaussian noise is denoted $w(t)$.

Our algorithms operate in discrete-time; sampling the received signal $r(t)$ at a rate of $M$ samples per symbol,
i.e., the sampling frequency $f_s=\frac{1}{T_s}=\frac{M}{T}$, yields 
\begin{equation}
    \begin{aligned}
      \label{eq:model}
      r_n = s_{n-p}Ae^{j\phi}e^{j2\pi\delta n}+w_{n},
    \end{aligned}
  \end{equation}
The continuous-time delay $\tau$ is decomposed into $\tau=pT_s+\Delta p$ with $pTs$ integer multiples of sample period and 
the fractional delay $\Delta p$ satisfying $-Ts/2 < \Delta p \leq T_s/2$. $\Delta p$ is neglected in this paper because 
our proposed algorithm proceeds at a very sufficient sample rate (in SDR). $\delta=f_d T_s$ is the normalized frequency offset. 
Besides that, we use $E_s/N_0$ to represent the ratio of signal energy to noise power spectral density (SNR).
Specifically, $E_s$ is the averaged symbol energy of the received signal over the length of the preamble.

% estension to reviewer 3, comment 3
Our goal of this paper is to analyze the complete signa\-l acquisition chain, which includes
the detection (time synchronization) and joint phase and frequency estimation of the preamble (carrier synchronization). 
The procedure is depicted in Figure~\ref{fig:sig_acquis_chain}.
The analysis of the signal acquisition chain will basically follow the steps 
that are shown in the figure as well as be implemented in software-defined radio (SDR).

\section{Detection and Time Synchronization}
\label{sec:detection}

As illustrated in Figure~\ref{fig:sig_acquis_chain}, the first step to analyze
the signal acquisition chain is to formulate the detector.
The detection algorithm proceeds sequentially and each step a window of length $N$ received samples with 
$N-1$ overlapped samples of the previous window is considered. The sequential detection problem solved 
in this paper is fundamentally equivalent to sequential frame synchronization~\cite{Massey_72,Lui_Tan_86,Scholtz_80}.
Since we mainly focus on the detection algorithm with respect to the timing instant (delay), to make analysis simpler,
the fractional delay of~\eqref{eq:model} is neglected by assuming a sufficient large sample rate in this section. 
The effect of fractional delay on detection will be discussed by looking at the receiver operating characteristics (ROC) in simulation (Section~\ref{sec:simulations}).

\begin{figure}[t]
  \centerline{\includegraphics[width=3.35in]{H1_H0_hypothesis.png}}
  \caption{Received sequence in observation window containing partial preamble (Blue: current received sequence. Red: received sequence at true delay $\bar{p}$)}
  \label{fig:H1_H0_hypothesis}
  \end{figure}

We start by looking at the likelihood ratio test (LRT) for detection task:
Let $H_0$ be the null hypothesis that the preamble is not completely presented in the received sequence from the obeservation window 
against the alternative $H_1$ that it does. 
Define $\Delta$ to be the distance between current received sequence at position $p_c$ and the true position ($\bar{p}$) of the preamble, 
i.e., $\Delta=|\bar{p}-p_c|$. Figure~\ref{fig:H1_H0_hypothesis} illustrates the case that the received sequence in observation window contains partial preamble.
It is symmetry if the partial preamble shows on the left side of the observation window ($p_c>\bar{p}$).
It is obvious that under $H_0$, $\Delta \neq 0$ while $H_1$ means $\Delta=0$. 
Based on~\eqref{eq:model}, the conditional likelihood ratio test (LRT) can be built 
between $H_0$ and $H_1$ by given the phasor $S{=}Ae^{j\phi}$ and frequency offset $b{=}e^{j2\pi \delta}$
at true delay $\bar{p}$ (under $H_1$),

\begin{equation}
    \label{eq:likelihood ratio}
    \begin{aligned}
    &\Lambda(R|S,b)=\frac{p_{R|H_1,S,b}(r|H_1,S,b)}{p_{R|H_0,S,b}(r|H_0,S,b)}= \\
    &\frac{\displaystyle \prod_{n=0}^{N-1}\frac{1}{\sqrt{\pi N_0}}\exp{(-\frac{1}{2}\frac{|r_{n}{-}s_nSb^n|^2}{N_0/2})}}
    {(\frac{1}{\sqrt{\pi N_0}})^N\displaystyle \prod_{n=\Delta}^{N-1}\exp{(-\frac{1}{2}\frac{|r_{n}{-}s_{n-\Delta}Sb^{n-\Delta}|^2}{N_0/2})} 
    \displaystyle \prod_{n=0}^{\Delta-1}\exp{(-\frac{1}{2}\frac{|r_n|^2}{N_0/2})}} \\
    &\LRT{H_1}{H_0} \eta.
    \end{aligned}
\end{equation}
Cancelling the common parts and taking the logarithm,~\eqref{eq:likelihood ratio} is reduced to

\begin{equation}
    \label{eq:log likelihood}
    \begin{aligned}
    \Re\{\sum_{n=0}^{N-1}r_{n}s^*_nS^*b^{-n}-
    \sum_{n=\Delta}^{N-1}r_{n}s^*_{n-\Delta}S^*b^{-(n-\Delta)}\}
    &\LRT{H_1}{H_0} \frac{N_0}{2}\ln\eta \\
    &+\frac{A^2}{2}\sum_{n=N-\Delta}^{N-1}|s_n|^2.
    \end{aligned}
\end{equation}
The second summation on the left hand side points to the inner product of the overlap between
the observed (partial) preamble and the true preamble. 
By plugging $r_{n}$ of hypothesis $H_1$, i.e., $r_{n}=s_nSb^n+w_n$, the summation simplifies 
to a function $\sigma$ in terms of $\Delta$

\begin{equation}
    \label{eq:partial ACF of the preamble}
    \sigma(\Delta)=A^2b^{\Delta}\sum_{n=\Delta}^{N-1}s_ns^*_{n-\Delta}+\sum_{n=\Delta}^{N-1}w_ns^*_{n-\Delta}S^*b^{-(n-\Delta)},
\end{equation}
which is Gaussian distributed with mean given  "partial" auto-correlation function (ACF) of the preamble at lag $\Delta$.
To quantify~\eqref{eq:partial ACF of the preamble}, the symbol sequence of the preamble can be chosen with a good autocorrelation property,
e.g., Gold sequence, that the partial ACF is approximately equal to zero for all $\Delta \neq 0$. 
% from here, it gives me a new and very important observation: (7) may explain why we obtain the estimator
% by assuming the known position of preamble (p_c=p) is meaningful. (7) and (8) tells us we can
% the detector works when plugping in the estimator by assuming the current position is the true position.
% then (6) -> (8).
By assuming the above, after some proper scaling of~\eqref{eq:log likelihood}, the LRT 
finally reduces to a generalized correlation function in terms of the time instant (delay) $p$,
\begin{equation}
    \label{eq:generalized_corr}
    \rho(p)=
    \frac{\Re\{\langle
      \bm{r}_{p},\hat{\bm{s}}_{p}\rangle\}}
    {||\bm{r}_{p}||\cdot||\hat{\bm{s}}_{p}||} \LRT{H_1}{H_0} \gamma
  \end{equation}
where $\hat{\bm{s}}_{p}$ represents the carrier-estimate corrected preamble, i.e., $\hat{s}_{p}[n]~{=}~s_{n}\hat{S}_{p}\hat{b}_{p}^n$ for $n=0,\ldots,N{-}1$,
at delay $p_c$. $\hat{b}_{p_c}$ and $\hat{S}_{p_c}$ are the corresponding frequency and phasor estimates at delay $p$.
$\bm{r}_{p}$ represents the received data sequence at delay $p$.
$\gamma$ is the normalized detection threshold which lies on the range of $[0,1]$.
Note, the LRT in~\eqref{eq:likelihood ratio} is not practical since the frequency and phasor offsets at true delay $\bar{p}$
is unknown while detection. Instead of LRT, a generalized likelihood ratio test (GLRT) based detector can be built relying on the frequency and phasor estimates from each window of received sequence;
Recall that the LRT of~\eqref{eq:likelihood ratio} is built conditionally on known frequency and phase offset 
at the true position of preamble.
Therefore, the estimation algorithm for $\hat{b}_{p}$ and $\hat{S}_{p}$ can also be derived by
assuming the current received sequence contains the true preamble (under $H_1$ hypothesis).
% this tells the reason why we do estimation by assuming the position of preamble is known.

In conclusion, a GLRT based detector of~\eqref{eq:generalized_corr} is derived for detection in this paper, and it relies on the frequency and phase estimates of the observed sequence.
Furthermore, since the sequential detection proceeds at every time instant, to make it work in practice, the complexity of the carrier estimates becomes
much crucial. A low computational-complexity estimator should be derived for real detection purpose.

%
% \subsection{Fractional Delay}
% don't know we should talk about the fractional delay or not for the time being. Just ignore it for a second.
% 

\section{Frequency and Phase Estimation}%
\label{sec:freq_est}

Frequency and phase estimation, also called carrier synchronization, is the next step after detection for coherent demodulation.
In this section, we discuss the estimation algorithm by assuming the time synchronization of the preamble is perfect, i.e., 
the observation window contains the complete preamble. For the same reason as in the detection section, we first derive the estimator
by assuming the effect of the fractional delay in signal model~\eqref{eq:model} is neglected.
In simulation section, we will discuss how much the value of fractional delay degrades the estimating accuracy of estimators by
comparing with different sampling rate.
% I forget to say in last meeting, the plot was truly obtained from a random sequence but with a good autocorrelation property.
% For this time, I compared with the reason with my previous sequence and gold sequence, the result shows the same.
% For figure 2, I think at 0 dB, since the SD estimator may not be accurate enough, the generalized correlation may not tend to be perfectly uncorrelated.
% For this reason, I just keep the previous results with some other modifications for this time.

For estimating frequency offset $\delta$ and the phasor $S=Ae^{j\phi}$, the maximum likelihood (ML) estimate of the parameters in~\eqref{eq:model} is given by

\begin{equation}
\label{eq:ML_f_S}
  \hat{\delta},\hat{S}=\min_{\delta,S=Ae^{j\phi}}\sum_{n=0}^{N-1}|r_n-s_nSe^{j2\pi\delta n}|^{2}.
\end{equation}
% extension to reviewer 1, comment 3
By taking the Wirtinger derivative with respect to $S$ and setting it equal to zero, a 
closed form for the estimated phasor $\hat{S}$ is readily derived,

\begin{equation}
    \label{eq:opt_S}
    \hat{S}=\frac{\sum_{n=0}^{N-1}{r_{n}s_n^{*}e^{-j2\pi\hat{\delta} n}}}{\sum_{n=0}^{N-1}|s_{n}|^2},
  \end{equation}
and $\hat{\phi}=\arg\{S\}$. We see the estimate of phasor $\hat{S}$ relies on the estimate of frequency $\hat{\delta}$.
It is shown later the derivation of $\hat{\delta}$ also plugs in the expression of phasor estimate in~\eqref{eq:opt_S}. Thus, the estimators for frequency and phasor are
joint estimators. Moreover, by plugging~\eqref{eq:opt_S} in~\eqref{eq:generalized_corr}, the GLRT based detector finally reduces to

\begin{equation}
  \label{eq:reduced_GLRT_detector}
  \rho(p)=
  \frac{|\hat{S}^{(\text{num})}_p|}
  {||\bm{r}_{p}||\cdot||\bm{s}||} \LRT{H_1}{H_0} \gamma
\end{equation}
where $\hat{S}^{(\text{num})}$ denotes the numerator of phasor estimate in~\eqref{eq:opt_S} and $||\bm{s}||$ is Euclidean norm of the preamble. From~\eqref{eq:reduced_GLRT_detector} to~\eqref{eq:generalized_corr}, the computational complexity is greatly decreased.

% extension to reviewer 1, comment 3, reviewer 3, comment 4
The frequency estimate is obtained similarly as the zero of the
derivative of~\eqref{eq:ML_f_S},

\begin{equation}
    \label{eq:intm_neces_cond1}
    \sum_{n=0}^{N-1}{(r_{n}s_n^{*}S^{*}ne^{-j2\pi \delta n}-s_ns_n^{*}n)=0}.
    \end{equation}
Note $\sum_{n=0}^{N-1}{s_ns_n^{*}n}$ is real
valued, which results in the imaginary part of left hand side of~\eqref{eq:intm_neces_cond1} be zero;
By plugging the estimate for $S$ of~\eqref{eq:opt_S} into~\eqref{eq:intm_neces_cond1} and rearranging the order of indexes, yields

\begin{equation}
    \label{eq:intm_neces_cond2}
    \Im\bigg\{\sum_{m=0}^{N-1}{\sum_{n=0}^{N-1}{nr_{n}r_{m}^{*}s_n^{*}s_me^{j2\pi \delta(m-n)}}}\bigg\} = 0.
  \end{equation}
A change of variables lets us focus on the difference between sampling instances $m$ and $n$.
With $k=m-n$,~\eqref{eq:intm_neces_cond2} becomes

\begin{equation}
    \label{eq:intm_neces_cond3}
    \Im\bigg\{\sum_{m=0}^{N-1}{\sum_{k{=}m-(N-1)}^{m}{(m{-}k)r_{m-k}r_{m}^{*}s_{m-k}^{*}s_me^{j2\pi \delta k}}}\bigg\}=0.
  \end{equation}
Reversing the order of summation in~\eqref{eq:intm_neces_cond3}, we get

\begin{equation}
    \begin{aligned}
    \label{eq:intm_neces_cond4}
    \Im\bigg\{&\sum_{k=-(N-1)}^{0}\sum_{m=0}^{N-1+k}{(m{-}k)r_{m-k}r_{m}^{*}s_{m-k}^{*}s_me^{j2\pi \delta k}+}\\
    &\sum_{k=1}^{N-1}\sum_{m=k}^{N-1}{(m{-}k)r_{m-k}r_{m}^{*}s_{m-k}^{*}s_me^{j2\pi \delta k}}\bigg\}= 0.
    \end{aligned}
  \end{equation}
The term for $k{=}0$ in~\eqref{eq:intm_neces_cond4} can be eliminated since it is real-valued. For $k \neq 0$, the positive and negative indices $k$ are symmetric. 
After grouping terms appropriately, the necessary condition for $\hat{\delta}$ is given by

\begin{equation}
    \label{eq:delta}
    J(\hat{\delta}) = \Im\bigg\{\sum_{k=1}^{N-1}{\sum_{m=k}^{N-1}{kr_{m-k}r_m^{*}s_{m-k}^{*}s_m}e^{j2\pi\hat{\delta}k}}\bigg\}=0.
    \end{equation}
This expression is fundamentally equivalent to conditions provided by Luise and Reggiannini~\cite{Luise_Reggiannini_95} and Fitz~\cite{Fitz_94}.
However,~\eqref{eq:delta} explicitly allows for pulse shaping and oversampling.

The estimator $\hat{\delta}$ in~\eqref{eq:delta} has no closed-form
solution.
In~\cite{Luise_Reggiannini_95}, it is approximated by replacing the exponential with its
Taylor series expansion.
In~\cite{Fitz_94}, an approximate solution is obtained via Euler's
identity for large $N$.
Both solutions have computational complexity $O(N^2)$ reflecting the
double summation.

Recall our paper focus on the joint detection and estimation problem. Thus, we propose a family of alternative solutions to~\eqref{eq:delta}.
A coarse solution with $O(N)$ complexity is used for operating at the sample
rate during the sequential GLRT detection;
it prioritizes low complexity at the expense of some loss of accuracy.
A fine solution is used to improve the estimation accuracy for coherent demodulation once the preamble has been detected.

\subsection{Solution I: (Coarse) Single-Difference (SD) Estimator}

The first estimator is rooted in the insight that at high SNR environment, every lag $k$ in~\eqref{eq:delta} can be used to
approximate the true frequency offset $\bar{\delta}$. Assume noise is very small, i.e.,
$r_m \approx s_mAe^{j(2\pi \bar{\delta} m+\phi)}$, and~\eqref{eq:delta} can be expanded to

\begin{equation}
    \label{eq:delta_extens_no_noise}
    \Im\bigg\{A^2\sum_{k=1}^{N-1}\sum_{m=k}^{N-1}k|s_{m-k}|^2|s_m|^2e^{j2\pi (\hat{\delta}-\bar{\delta})k}\bigg\}=0.
    \end{equation}
Note that in~\eqref{eq:delta_extens_no_noise} the inner summation is purely real for every lag~$k$ if $\hat{\delta}=\bar{\delta}$.
This observation suggests that an unbiased estimate of the frequency offset can be obtained by using only a single lag~$k$
from~\eqref{eq:delta}. The approach lowers the complexity from $O(N^2)$ to $O(N)$ and permits a closed-form solution for $\hat{\delta}$.  
The disadvantage of the estimator by using single lag $k$ is its insufficient estimating accuracy since it lacks of the processing gain by outer integrator.
Thus, the estimator is named as the coarse SD estimator. We just omit "coarse" for simplicity from now.
However, due to the low complexity, the SD estimator can be used for 
frequency offset correction of sequential GLRT detector in~\eqref{eq:generalized_corr}. 

\subsubsection{Closed-form expression} 
For one lag $k$, the primary SD estimator is directly obtained by reducing~\eqref{eq:delta_extens_no_noise},
\begin{equation}
    \label{eq:delta_SD}
    \est{\delta}{\text{primary}-\sd}(k)=-\frac{\arg\big\{\sum_{m=k}^{N-1}r_{m-k}r_m^*s_{m-k}^*s_m\big\}}{2\pi k}.
\end{equation}

\subsubsection{Performance of primary SD estimator}
In low (or moderate) SNR environment, i.e., noise effect cannot be ignored, the argument of numerator, denoted as $W(k)$, in~\eqref{eq:delta_SD} can be extended to 

\begin{equation}
  \label{eq:delta_extens_w_noise}
  \begin{aligned}
    W(k)&=\sum_{m=k}^{N-1}r_{m-k}r_m^*s_{m-k}^*s_m= \sum_{m=k}^{N-1} \Big( A^2|s_{m-k}|^2|s_m|^2e^{-j2\pi \bar{\delta} k} + \\
    &w_m^* S|s_{m-k}|^2s_m e^{j2\pi \bar{\delta}(m-k)} + w_{m-k}S^*|s_m|^2s_{m-k}^* e^{-j2\pi \bar{\delta} m} + \\
    &w_{m-k}w_m^*s_{m-k}^*s_m \Big) .
  \end{aligned}
\end{equation}

To interpret~\eqref{eq:delta_extens_w_noise}, recognize that the first term of right hand side is
deterministic and provides the mean of the expression. The two middle terms yield a zero-mean, complex Gaussian random variable. 
The last term performs another zero mean-random variable with a second kind Bessel distribution, which is close to Gaussian distribution. Moreover, we assume
$N{-}k$ is large, by central limit theorem, the second kind Bessel random variable is considered approximately Gaussian.      
Thus,~\eqref{eq:delta_extens_w_noise} is summarized as
% ask for more details are lack of. 

\begin{equation}
  \begin{aligned}
    \label{eq:ori_pdf_W}
    W(k) \sim \cn\bigg(\Big(
    \frac{N{-}k}{A^2}\Big){\Big(\frac{E_s}{M}\Big)}^2e^{-j2\pi \bar{\delta} k},
    2&\Big(\frac{N{-}k}{A^4}\Big)\frac{N_0}{2}{\Big(\frac{E_s}{M}\Big)}^3+ \\
    &\Big(\frac{N{-}k}{A^4}\Big)\Big(\frac{N_0}{2}\Big)^2{\Big(\frac{E_s}{M}\Big)}^2\bigg).
  \end{aligned}
\end{equation}
Here $A^2|s_m|^2 {\approx} E_s/M$ denotes the average energy per
sample. To discuss the performance of $W(k)$ with respect to the SNR, we look at the ratio of 
(absolute) square of mean to variance in~\eqref{eq:ori_pdf_W}, which yields the relationship between
output SNR of $W(k)$ and input SNR,

\begin{equation}
  \begin{aligned}
    \label{eq:SNR_out}
    \text{SNR}_{\text{out}}=\frac{|\mu_{W(k)}|^2}{\sigma^2_{W(k)}} 
    =&~\frac{N-k}{\displaystyle 2\cdot\frac{N_0}{2}/\frac{E_s}{M}+\Big(\frac{N_0}{2}\Big)^2/\Big(\frac{E_s}{M}\Big)^2} \\
    &~~~~~~~~~\quad =\frac{N-k}{2/\text{SNR}_{\text{in}}+1/\text{SNR}_{\text{in}}^2}
  \end{aligned}
\end{equation}
where $\mu_{w(k)}$ and $\sigma^2_{W(k)}$ deonte the mean and variance of $W(k)$ provided in~\eqref{eq:ori_pdf_W}, and $\text{SNR}_{\text{in}}$ is defined as the input average sample energy to noise power spectral density, specifically, $\frac{E_s}{M}/\frac{N_0}{2}$. In~\eqref{eq:SNR_out}, 
$N-k$, which is the size of integrator in~\eqref{eq:delta_extens_w_noise}, exactly provides the processing gain; At relatively high SNR,
the inverse of squared input SNR, which is from the variance of second kind Bessel random variable, can be neglected. Then $\text{SNR}_{\text{out}}$
exhibits a linear relationship with $\text{SNR}_{\text{in}}$. On the other hand, at low input SNR, the inverse squared SNR dominates and degrades the $\text{SNR}_{\text{out}}$
of $W(k)$ quadratically, which makes the $\text{SNR}_{\text{out}}$ fast approach and then be lower than the minimum requirement that maintains the required accuracy of 
primary SD estimator for building detector and later a fine estimator.

Recall from~\eqref{eq:delta_SD}, the primary SD estimator $\hat{\delta}_{\text{primary}-\text{SD}}(k)$ requires $\arg\{W(k)\}$.
The full probability density function (pdf) of $\arg\{W(k)\}$ is derived in the appendix \ref{AL},
where it is also shown that a good approximation, valid for moderate SNR, 
is Gaussian. Specifically 

\begin{equation}
    \label{eq:sol_pdf_W}
    \arg\{W(k)\} \sim \n\bigg(\angle \mu_{W(k)},\frac{\sigma^2_{W(k)}}{|\mu_{W(k)}|^2}\bigg).
  \end{equation}
By plugging~\eqref{eq:ori_pdf_W} without adding second Bessel variance term, and~\eqref{eq:sol_pdf_W} into~\eqref{eq:delta_SD}, the primary SD estimator
is approximately Gaussian distributed at moderate SNR with pdf

\begin{equation}
    \label{eq:pdf_delta}
        \est{\delta}{\text{primary}-\sd}(k) \sim \n \bigg(\bar{\delta},\frac{M}{4\pi^2k^2(N{-}k)E_s/N_0}\bigg).
  \end{equation}  
We see that $\hat{\delta}_{\text{primary}-\text{SD}}(k)$ is unbiased. 
Note, the distribution of the primary SD estimator in~\eqref{eq:pdf_delta} only exploits a good fit at moderate SNR because of
the constraint in~\eqref{eq:sol_pdf_W}. A rough calculation of the distribution of primary SD estimator at low SNR is by simply 
replacing the variance of Gaussian variable with the variance of second kind Bessel
of~\eqref{eq:ori_pdf_W} and plugging in~\eqref{eq:sol_pdf_W}. The resulting variance, or equavilently, mean-squared error (MSE) of the primary 
SD estimator at low SNR, compared to high SNR, is increased by $M/(4E_s/N_0)$.
For example, when $E_s/N_0=0.3$, i.e., \numb{-5}\dB, with a normal oversampling factor $M=4$,
the performance of primary SD estimator is approximately degraded by a factor of 3.3.
The solution to mitigate the impairment of primary SD estimator at low SNR
is given in the next section.

\subsubsection{General SD estimator}

\begin{figure}[t]
  \centerline{\includegraphics[width=3.4in]{general_SD_estimator.png}}
  \caption{Flow diagram of implementing an alternative general SD estimator}
  \label{fig:general_SD_estimator}
  \end{figure}

The reason for primary SD est-imator in~\eqref{eq:delta_SD} has a relatively bad performance at low SNR is due to the large noise
disturbance that greatly decorrelates the received signal and reference signal at every sample instant.
An alternative SD estimator is proposed by changing the order of computation in~\eqref{eq:delta_SD}.
Specifically, instead of calculating the correlation at each time instant and then averaging, we first do averaging of 
correlations for coherent time instants; Then, use the remaining processing gain to average 
correlations between non-coherent time instants.
The process is illustrated in Figure~\ref{fig:general_SD_estimator}. 


It should be noted that, to prevent "aliasing" of the frequency estimate, it's also required to choose the lag $k$ small enough to ensure $2\pi |\bar{\delta}| k < \pi$.
Assuming an upper bound $\delta_{\text{max}}$ on the normalized frequency offset is available, 
the optimal choosing policy for lag $k_{\opt}$ is refined as

\begin{equation}
    \label{eq:est_k_opt}
    k_{\opt}=\left\{
      \begin{array}{cl}
        \lfloor\frac{2}{3}N\rfloor
        & \text{for $\frac{1}{2\delta_{\max}}>\lfloor\frac{2}{3}N\rfloor$} \\
        \lceil\frac{1}{2\delta_{\max}}-1\rceil
        & \text{for $\frac{1}{2\delta_{\max}} \leq\lfloor\frac{2}{3}N\rfloor$},
      \end{array}
    \right.
  \end{equation}
where $\lfloor \cdot \rfloor$ and $\lceil \cdot \rceil$ are the floor and ceil operation, respectively.

% \subsubsection{Low-SNR Improvement}
% Recall that the expression of SD estimator with optimal choice $k$ only fits a moderate SNR.
% The performance of SD at low SNR is also crucial and needs to be discussed.
% One way to improving the accuracy is by averaging $K$ estimates of SD with different lags $k$.
% We call the resulting estimator the $K-$SD estimator, $\est{\delta}{K\text{-SD}}$.
% In this case, we trade off a $K$-fold increase in computational complexity for lower variance.

% Let $\bm{u}$ be a vector of non-negative, with
% $\sum_{k \in {\cal K}}u_k=1$; here ${\cal K}$ represents the set of $K$
% lags to be averaged.
% % extension to reviewer 1, comment 5, reviewer 3, comment 4
% Simple linear combining of $K$ SD estimators yields the $K-$SD estimator

% \begin{equation}
%   \label{eq:K_SD_est}
%       \est{\delta}{K\text{-SD}}=\sum_{k \in \cal K}\est{\delta}{\sd}(k)u_k.
% \end{equation}
% The optimal weight vector $\bm{u}_\opt$ can be obtained by minimizing the variance of $\est{\delta}{K\text{-SD}}$,
% which is well-known as

% \begin{equation}
%   \label{eq:u_opt}
%       \bm{u}_\opt=\frac{\bm{C}^{-1}\bm{1}}{\bm{1}^T\bm{C}^{-1}\bm{1}}.
% \end{equation}
% $\bm{C}$ is the autocovariance matrix between the $K$ SD estimators and $\bm{1}$ represents the column vector of one. 
% Unfortunately, it is generally difficult to know the full information of $\bm{C}$. However,
% If the lags $k \in {\cal K}$ are chosen to satisfy the spacing between any pair is at least
% equal to the oversampling factor $M$, then the estimates to be combined are approximately uncorrelated and unbiased; The  
% resulting $\bm{C}$ is a diagonal matrix of variance of each SD estimator. For example, a good choice for selecting 3 lags is 
% ${\cal K}=\{k_\opt-M,k_\opt,k_\opt+M\}$. The optimal weights $\bm{u}_\opt$ are proportional to the inverse of the
% variances in~\eqref{eq:pdf_delta}.

\subsection{Solution II: Newton-Method (NM) Estimator}

The SD estimator emphasizes low-complexity property and is intended to provide merely sufficiently good carrier synchronization
to enable coherent detection. Once the signal has been acquired, the SD estimator can be improved by 
investing additional computations. Since detection events are rare, the computational complexity is of little concern.

The principle is to use the SD (or $K-$SD) estimator as the starting point for a Newton-type iteration 
aimed at finding a better solution to the necessary condition~\eqref{eq:delta}. 
In principle, multiple iterations are possible to produce successively better approximations to the root of
$\J(\hat{\delta})$ in~\eqref{eq:delta}. Specifically, the iterations are given by

\begin{equation}
    \label{eq:iter_NM_est}
    \est{\delta}{\nm}^{(i+1)}=\est{\delta}{\nm}^{(i)}-
    \frac{\J(\est{\delta}{\nm}^{(i)})}{\J^\prime(\est{\delta}{\nm}^{(i)})}
  \end{equation}
where $\est{\delta}{\nm}^{(0)}~{=}~\est{\delta}{\sd}(k_{\opt})$ is the starting point of the iteration and
$\J^\prime(\cdot)$ denotes the derivative of $\J$ with respect to $\hat{\delta}$. Specifically,

\begin{equation}
    \label{eq:derivative of delta}
    J^\prime(\hat{\delta}) = \Im\bigg\{\sum_{k=1}^{N-1}{\sum_{m=k}^{N-1}{j2\pi k^2r_{m-k}r_m^{*}s_{m-k}^{*}s_m}e^{j2\pi\hat{\delta}k}}\bigg\}.
    \end{equation}
Our simulations indicate that only a single iteration is usually sufficient to achieve very good accuracy.




\section{Simulation Results}%
\label{sec:simulations}

The simulation section illustrates the results in the order of detection, estimation and joint detection and estimation.
Four different lengths of reference symbol sequences are simulated with $50\%$ rolloff Square-Root Raised Cosine (SRRC) pulses.
The reference sample sequence is chosen from Gold sequence and modulated by a QPSK alphabet with good autocorrelation properties.
We assume the normalized frequency offset is small enough so that the design parameter $k$ of the SD estimator
can be chosen optimally by $2/3$ length of the preamble (For $K-$SD estimator, all the estimates of SD satisfying the aliasing limits). 
% may give more sentence here.

\subsection{Simulation Results for Detection}

\begin{figure}[t]
    \centerline{\includegraphics[width=3.4in]{generalized_correlation.png}}
    \caption{Performance of GLRT detector of~\eqref{eq:generalized_corr} at each delay of received stream (SNR$=\numb{0}\dB$, no fractional delay, dashed line: the position of $\bar{p}$)}
    \label{fig:Generalized correlation}
    \end{figure}

\begin{figure}[t]
    \centerline{\includegraphics[width=3.4in]{receiver_operating_characteristics.png}}
    \caption{Receiver operating characteristics of GLRT based detector for different lengths of reference sequence and SNR}
    \label{fig:Receiver operating characteristics}
    \end{figure}

\begin{figure}[t]
    \centerline{\includegraphics[width=3.4in]{false_alarm_and_detection_probability.png}}
    \caption{False alarm and detection ratio for different SNR and sizes of preamble ($M=4$. Blue: false alarm probability. Red: detection probability)}
    \label{fig:False alarm and detection}
    \end{figure}

\begin{figure}[t]
    \centerline{\includegraphics[width=3.4in]{false_alarm_and_detection_probability_fractional_delay.png}}
    \caption{Comparison between false alarm and detection ratio with and without maximum $\Delta p$ for same $L_0$ and different SNR ($M=4$. Blue: $P_{\text{FA}}$. Red: $P_{\text{D}}$)}
    \label{fig:False alarm and detection with frac delay}
    \end{figure}

In this section, we first illustrate the simulation results of the detection algorithm without considering the fractional delay.
The effect of fractional delay on detection will be discussed at the end of this section. 

Figure~\ref{fig:Generalized correlation} illustrates the performance of the proposed detector in~\eqref{eq:generalized_corr} without fractional delay for 
$L_0=32$, $M=8$,~\numb{0}\dB~SNR. The GLRT based detector performs robust at low SNR since it achieves the highest generalized correlation at $\bar{p}$.
However, from the adjacent correlations centered at $\bar{p}$ that we point out,
the correlation decays very slow and thus it will make much more challenging to choose the threshold to distinguish between 
the true position of received sequence and its corresponding adjacent positions. This is due to pulse shaping and oversampling. Specifically,
the width of "mainlobe" depends on the shape of pulse and the value of $M$, e.g., 
it will become wider if $M$ is larger. 
To accommodate this imperfect autocorrelation influence by pulse shaping and oversampling, we adjust the detection algorithm by finding the local maximum of the correlation near $\bar{p}$ instead of
just comparing the correlations with threshold to make the decisions at each delay. 

Figure~\ref{fig:Receiver operating characteristics} shows the receiver operating characteristics (ROC) of the proposed (adjusted) detector.
It basically illustrates that the detector has good ROCs at very low SNR. Specifically, the detector achieves perfect ROCs for positive SNR for all length of $L_0$; 
Moreover, it is still robust to negative SNR environment. For instants, we can select such a threshold to achieve  
$P_{\text{FA}} \approx 10^{-3}$ and $P_{\text{D}} \geq 0.8$ simultaneously when $L_0=32$ and SNR $=$~\numb{-2}\dB.
However, it is not clear from this figure to pick an exact threshold for certain detection purpose.

Figure~\ref{fig:False alarm and detection} shows the performance of detector but from another perspective. Specifically, it gives us a deep insight into
the false alarm and detection probabilities versus the threshold. 
Basically, we have two observations. First, compared with the two curves of~\numb{0}dB~and~\numb{10}dB SNR for $L_0=32$,
we see the latter needs to achieve the same false alarm probability with a higher threshold. This is because the level of "sidelobe" in Figure~\ref{fig:Generalized correlation} is
increasing in terms of SNR; If we look at the same two curves of $P_{\text{D}}$, the curve of a high SNR maintains a perfect $P_{\text{D}}$ at a higher threshold. This is due to
the increasing level of "mainlobe" via SNR. Second, compared with two curves with different $L_0$ and same SNR, we see that the received sequence with more symbols achieves the same $P_{\text{FA}}$
at lower threshold. The reason is that the sequence with more symbols will have a better autocorrelation property.  
Moreover, it is commonly hard to pick such a threshold to accommodate all specific detection purposes.
For simulation purpose, now if we want to determine a threshold to work for all reference sequences with $L_0 {\geq} 32$ and SNR up to~\numb{10}\dB,
based on Figure~\ref{fig:False alarm and detection} and Neyman-Pearson criterion, $\gamma=0.43$ is a good choice to meet $P_{\text{FA}}<1e^{-3}$ and nearly perfect $P_{\text{D}}$ for all lengths of $L_0$.

To this point, we discuss the effect of fractional delay $\Delta p$ in model~\eqref{eq:model} on modified detection algorithm. 
Above, we assume $\Delta p$ is neglected when sampling rate $f_s$ is sufficiently large. 
However, when $f_s$ is not much bigger than symbol rate, 
or equivalently, the oversampling factor $M$ is relatively small, 
the value of $\Delta p$ will degrade the performance of detection and estimation due to
mismatching between the preamble in received and reference sequence.
For example, when $M{=}1$ and the preamble has the maximum fractional delay 
equaling to $\Delta p=\pm\frac{1}{2f_s}=\pm\frac{T}{2M}=\pm\frac{T}{2}$, 
half of the preamble in received and reference sequence is mismatched thus the peak of
generalized correlation in Figure~\ref{fig:Generalized correlation} will disappear.
Instead, Two peaks will show up: one at $\bar{p}$ and another one at the adjacent $p$ depending on the sign of $\Delta p$.
Recall, the detection algorithm is to find the local maximum of generalized correlation around $\bar{p}$.
Thus, there is no doubt that $\Delta p$ will greatly decrease the detection probability, as shown in Figure~\ref{fig:False alarm and detection with frac delay}.

In Figure~\ref{fig:False alarm and detection with frac delay}, the most intuitive observation is that the curves of $P_{\text{D}}$ with (maximum) fractional delay for all SNR
do not start at 1 when $\gamma$ is very low. The detector is hard to make the correct decision between the adjacent two delays with similar high correlations. 
If we compare with the two curves of $P_{\text{D}}$ for SNR $=\numb{0}\dB$ and $\numb{10}\dB$
both with fractional delay, the initial $P_{\text{D}}$ of $\numb{10}\dB$ SNR is approxmiately 0.6 and greater than 0.5 at $\numb{0}\dB$.
This is because at low SNR, the noise level degrades the detection performance; While, at high SNR, 
the percentage of the mismatched preamble between received and reference sequence 
is below $50\%$ since $\Delta p{=}{\pm}\frac{T}{2M}{=}{\pm}\frac{T}{8}$. The generalized correlation
at $\bar{p}$ will be larger than at the adjacent delay with the probability more than $1/2$.

We can also see the fractional delay has a small effect on false alarm probability, which is important. For example, as we discussed above,
if we want to choose a threshold to work for the detection task that all reference sequences with $L_0 {\geq} 32$
and SNR${\leq}$\numb{10}\dB, then from Figure~\ref{fig:False alarm and detection} and~\ref{fig:False alarm and detection with frac delay},
$\gamma=0.43$ is still a good choice to meet the requirement of false alarm probability. Note, when $\Delta p$ is large, although $\Delta p$ will "lead" the 
detector to make the decision between the two adjacent delays with similar highest generalized correlation,
we can just keep either one as the detection result but care more about the estimation accuracy with the determined received sequence
for coherent demodulation. Thus, in the next section, we will talk about the degradation of estimation accuracy with different levels of $\Delta p$.
The goal is to choose the oversampling factor $M$ that makes the estimation error negligible.

\subsection{Simulation Results for Estimation}

\begin{figure}[t]
    \centerline{\includegraphics[width=3.4in]{accuracy_NM_SD.png}}
    \caption{Accuracy of the SD and SD (or $K$-SD) based NM estimators ($L_0=32$)}
    \label{fig:accuracy_NM_SD}
    \end{figure}

\begin{figure}[t]
    \centerline{\includegraphics[width=3.4in]{accuracy_NM_traditional.png}}
    \caption{Accuracy of the NM estimator and conventional estimators ($L_0=32$)}
    \label{fig:accuracy_NM_traditional}
    \end{figure}

\begin{figure}[t]
    \centerline{\includegraphics[width=3.4in]{accuracy_NM_with_M_frac.png}}
    \caption{Accuracy of NM estimator with maximum fraction delay for different value of oversampling factor ($L_0=32$, $\Delta p=\frac{T}{2M}$)}
    \label{fig:accuracy_NM_with_M_frac}
    \end{figure}

\begin{figure}[t]
    \centerline{\includegraphics[width=3.4in]{freq_NM_with_different_size.png}}
    \caption{Accuracy of NM frequency estimate in joint detection and estimation ($\gamma=0.43$, $M=4$, $\Delta p \in (-\frac{T}{8},\frac{T}{8}]$)}
    \label{fig:accuracy_freq_NM_joint}
    \end{figure}

\begin{figure}[t]
    \centerline{\includegraphics[width=3.4in]{phi_NM_with_different_size.png}}
    \caption{Accuracy of the NM phase estimate in joint detection and estimation ($\gamma=0.43$, $M=4$, $\Delta p \in (-\frac{T}{8},\frac{T}{8}]$)}
    \label{fig:accuracy_phi_NM_joint}
    \end{figure}

In this section, we focus on the estimating accuracy of our proposed estimators (SD, or $K-$SD and NM)
by assuming no fractional delay and the sequential GLRT detection progress is perfect.
The simulation shows the NM estimator can achieve as good accuracy as those conventional estimators in ~\cite{Kay_89,Luise_Reggiannini_95,Fitz_94}
for coherent demodulation at moderate SNR. 
Then, the effect of fractional delay $\Delta p$ on estimating accuracy with respect to $M$ will be discussed.
We find that $M=4$ yields a negligible estimation error from $\Delta p$.
At the end of this section, we will also show the estimating accuracy of NM estimator after joint detection and simulation for coherent demodulation.  

Figure~\ref{fig:accuracy_NM_SD} gives the insight into the accuracy of two proposed estimators.
The lower dashed line denotes the Cramer-Rao vec-tor bound (CRVB) for frequency estimate in~\eqref{eq:model}
multiplied by $M$, which is given by

\begin{equation}
    \label{eq:CRVB_freql}
    \text{CRVB}(M\delta) \geq \frac{3}{2\pi^{2}L_{0}^3E_s/N_{0}}.
  \end{equation}
Furthermore, the CRVB for $\phi$ is derived as

\begin{equation}
    \label{eq:CRVB_phi}
    \text{CRVB}(\phi) \geq \frac{2}{L_{0}E_s/N_{0}}.
  \end{equation}
The derivation steps of~\eqref{eq:CRVB_freql} and~\eqref{eq:CRVB_phi} are given in Appendix~\ref{BL}.
In Figure~\ref{fig:accuracy_NM_SD}, we see the NM estimator approaches the CRVB at SNR$=\numb{4}\dB$. 
Note, the accuracy of the SD estimator is crucial
not only just for building the GLRT detector but deciding the accuracy of the NM estimator.
The evidence is that the NM estimator performs even worse than the SD estimator at low SNR.
This is because the Newton iteration of~\eqref{eq:iter_NM_est} converges occasionally to local
minimum away from the true frequency offset if the initial (SD) estimate is far from the true frequency offset. 
Moreover, Figure~\ref{fig:accuracy_NM_SD} shows the averaging method of~\eqref{eq:K_SD_est}
improves the accuracy of the SD estimator thus improves the NM estimator at low SNR.

Figure~\ref{fig:accuracy_NM_traditional} compares the performance of the NM estimator and conventional estimators in ~\cite{Kay_89,Luise_Reggiannini_95,Fitz_94}.
Related to Figure~\ref{fig:accuracy_NM_SD}, we co-nclude that the NM estimator can achieve as good accuracy as those autocorrelation-based estimators 
(L\&R estimator~\cite{Luise_Reggiannini_95} and Fitz estimator~\cite{Fitz_94})
when the SD estimator is accurate enough. Put it another way, without increasing complexity by averaging with multiple SD estimators, 
the NM estimator can achieve the same good accuracy as the traditional estimators at moderate SNR.

Now we are going to discuss the effect of $\Delta p$ on estimating accuracy of the NM estimator.
In the previous section, we got the conclusion that $\Delta p$ both degrades the detection probability
and estimating accuracy because of the mismatching between the preamble in received and reference sequence.
The solution to dealing with the degradation of detection is to maintain the decision of detector but sacrifice the 
accuracy of the estimator. Thus, we need to select the oversampling factor $M$ that makes the fractional delay $\Delta p$ small enough and the estimation error negligible.  
Figure~\ref{fig:accuracy_NM_with_M_frac} illustrates the estimating accuracy of NM estimator with maximum fractional delay
equaling to $\Delta p{=}\frac{T}{2M}$ for different $M$. It can be seen that when $M=1,2$, i.e., $\Delta p=\frac{T}{2},\frac{T}{4}$, the gap between
the mean-squared error (MSE) and the CRVB is obvious; The MSE of two curves for $M=4$ and $M=8$ both nearly approach the
CRVB. But note, the reference sequence with $M=8$ induces an extra double computational complexity than $M=4$. Thus, $M=4$ is an ideal choice
for joint detection and estimation purpose in this paper.

At the end of simulation section, the complete signal acquisition chain (joint detection and estimation) is simulated.
Figure~\ref{fig:accuracy_freq_NM_joint} and~\ref{fig:accuracy_phi_NM_joint} illustrate the performance of NM estimator after joint detection and 
estimation. $\Delta p$ is included with the range determined by $M$.
The parameters, e.g., the value of $\gamma$, the oversampling factor $M$, are chosen based on the previous discussion.
No averaged SD estimator is used because we want to reduce the complexity of the sequential detection process.
Compared with Figure~\ref{fig:False alarm and detection} and Figure~\ref{fig:accuracy_NM_SD}, the reason for the NM estimator not approaching the CRVB at low SNR
is due to non-sufficient accuracy of the SD estimator. Besides that, we can get a common conclusion that by increasing
the size of the preamble, better performance of carrier synchronization is realized and CRVB is approached at lower SNR.

\begin{figure*}[t]
    \centerline{\includegraphics[width=6.5in]{SDR_receiver.png}}
    \caption{Block diagram for implementing the proposed joint detection and estimation algorithm in Threading Building Blocks}
    \label{fig:SDR_receiver}
    \end{figure*}

\begin{figure*}[t]
    \centerline{\includegraphics[width=6.5in]{SDR_phasor_SD.png}}
    \caption{Block diagram for modified computation of phasor estimator (the middle level of dashed area in Figure~\ref{fig:SDR_receiver}) in Threading Building Blocks}
    \label{fig:SDR_phasor_SD}
    \end{figure*}

\section{Implementation on Software-defined radio}
\label{sec:real_implementation}

Two main aspects of measuring the performance of modern communcation systems, except for accuracy,
are the latency and throughput. In the previous sections, we focus on explaining and showing
how much accuracy of our proposed joint detection and estimation algorithm can be.
In order to realize it on software-defined radio (SDR), the algorithm, especially the detection algorithm, should be refined
to fitting very high sample rate since it is applied on every time instant.

\subsection{Receiver Side (algorithm)}

Figure~\ref{fig:SDR_receiver} illustrates some detailed changes of our proposed algorithm.
First, the style of the observation window with feedback loop in Figure~\ref{fig:sig_acquis_chain} is eliminated
because it incurs a very large amount of overhead. Instead, the received data is directly buffered into 
a fixed size (normally larger than the length of the preamble) of frame without overlapping. Now
the problem may happen when the preamble in the received data is cut off over two frames. 
The shift register is reused after the buffer to solve this problem. Compared with the old method, the received data packed in frames can be processed in order without
waiting for the result of detecting; Moreover, the length of the shift register can be also reduced. 
For example, if we look at formula of the SD estimator in~\eqref{eq:delta_SD}, the length of the shift register can be shortened from $N$ to $k+1$,
i.e., if $k=\frac{2}{3}N$, the overhead is approximately reduced by $\frac{N}{3}$.

Second, some key steps of the proposed detection algorithm should be computed more efficiently.
For example, to calculate the frequency estimate of SD estimator,~\eqref{eq:delta_SD} can be intepreted as the convolution between autocorrelation vector of received samples and of the preamble (one of them should be reversed). 
Thus, the most efficient way of calculating~\eqref{eq:delta_SD} could be the fast fourier transform (FFT). 

Another important improvement that should be emphasized is about calculating the phase (phasor) estimate
of the SD estimator in~\eqref{eq:opt_S}. No
te,~\eqref{eq:opt_S} is a time-varying convolution, which cannot be computed by FFT.
To approximate the dot product in~\eqref{eq:opt_S}, we re-order the computation by first calculating the dot product between the (partial) received data vector ($\bm{r}$) and the (partial) preamble ($\bm{s}$);
the result is then corrected (multiplied) by the frequency estimate at the middle position of data vector and added together. Specifically, the numerator of

\begin{equation}
    \label{eq:refined_opt_S}
    \hat{S} \approx \sum_{i=0}^{L_2-1} \sum_{m=iL_1/L_2}^{(i+1)L_1/L_2-1}
    \sum_{n=mN/L_1}^{(m+1)N/L_1-1}r_ns_n^* 
    e^{-j\pi \hat{\delta}\frac{N(2m+1)}{L_1}}.
  \end{equation}
As illustrated in Figure~\ref{fig:SDR_receiver}, we compute~\eqref{eq:refined_opt_S} in two stages, each
stage contains $L_1$ (or $L_2$) parallel sub cells. 
Thus, we achieve the functional parallelism by an approximate of $L_1 \cdot L_2$ speed up.
To get a good approximation, $L_1$ should not be too small and 
$L_2$ is the factor of $L_1$.

\subsection{Signal Transmission Path}

In this section, we briefly talk about the signal transmitting path, which is used 
for testing our algorithm. The hardware connection is fairly easy. 
Each of two processors (computers) connected to a universal software radio peripheral (USRP) 
by one 5-Gigabit Ethernet cable as the transmitter or receiver. Between two USRPs,
a cable with~\numb{30}\dB~attenuation connects the RX/TX port and RX port.
At the transmitter side, the processor needs to tell the (transmitter) USRP the sample rate, 
the baseband signal and frequency, the carrier frequency, the transmitter gain, etc. 
Then, the USRP transmits the analog signal to the (receiver) USRP through the two ports.
At the receiver side, the received analog RF signal is first down-converted to baseband, down-sampled to 
discrete-time data stream and finally stored in the local network. After that, our proposed algorithm as illustrated
in Figure~\ref{fig:SDR_receiver} can be tested by requesting the data from the local network.

\subsection{Performance of SDR}

To measure the performance of the SDR, we focus on the accuracy, throughput and latency of our proposed algorithm.
Some parameters in Figure~\ref{fig:SDR_receiver} should be chosen by the following rules:
First, the length of the preamble should be chosen short enough to get the largest throughput as long as it does not degrade the accuracy;
Second, the length of frame should be chosen as the power of 2 to achieve the best performance of FFT; Moreover, 
it trades off between the overhead and latency, e.g., short frame has small latency but large overhead. 
Third, as discussed above, $L_1$ in~\eqref{eq:refined_opt_S} should not be chosen too small for a good approximation of phasor estimate. However, the large number of 
parallel executions will occupy the most threads of the computer.

As tested, such parameters are determined that can achieve a relatively good performance: 
The preamble is chosen with the number of symbols $L_0=32$ and oversampling factor $M=4$.
the length of the frame is determined by 8192; Four complete preambles are embedded in each frame. 
The number of sub cells for two stages are $L_1=2L_2=16$. Furthermore, some parameters of USRP are set as follows:
The sample rate at transmitter is 10 MS/s; The transmitter gain plus receiver gain is~\numb{20}\dB.
Based on above, we get the maximum throughput is around 4.5 MS/s and the latency is near 1 ms;
The detection algorithm is very robust and the false alarm probability is near 0.




% % extension to reviewer 1, comment 8
% \input{conclusion} can be finished later

% extension to reviewer 1, comment 6, reviewer 3, comment 4
\begin{appendices}

\section{Derivation of distribution for the SD estimator, proof of (\ref{eq:sol_pdf_W})}
\label{AL}

To prove the distribution of $\arg\{W(k)\}$, where $W(k)$ is a complex Gaussian 
random variable, we assume $W=X+jY$, $X$ and $Y$ are two Gaussian random 
variables with distributions $X {\sim} \n(\mu_x,\sigma^2)$ 
and $Y {\sim} \n(\mu_y,\sigma^2)$. Here, $W$, $X$, $Y$, $\mu_x$, $\mu_y$ 
and $\sigma^2$ all depends on $k$, we just write those for notation simplicity. 
We further assume $\mu_w$ to be the mean of $W$. The probability density function 
of $W$ is given by

\begin{equation}
    \label{eq:AL1}
    \begin{aligned}
    &f_W(w) \\
    &{=}f_{X,Y}(x,y){=}\frac{1}{2\pi \sigma^2}\exp\bigg({-}\frac{(x-\mu_x)^2+(y-\mu_y)^2}{2\sigma^2}\bigg).
    \end{aligned}
\end{equation}
Let $x=r\cos\theta$, $y=r\sin\theta$,~\eqref{eq:AL1} can be transformed into polar coordinate,

\begin{equation}
    \label{eq:AL2}
    \begin{aligned}
    &f_W(w){=} f_{R,\Theta}(r,\theta) \\
    &{=}\frac{r}{2\pi \sigma^2}\exp\bigg({-}\frac{(r\cos\theta-\mu_x)^2+(r\sin\theta-\mu_y)^2}{2\sigma^2}\bigg) \\
    &{=}\frac{r}{2\pi \sigma^2}\exp\bigg({-}\frac{r^2{+}\mu_x^2{+}\mu_y^2}{2\sigma^2}\bigg)\exp\bigg(\frac{r}{\sigma^2}(\mu_x\cos\theta{+}\mu_y\sin\theta)\bigg)
    \end{aligned}
\end{equation}
Plugging $\mu_x=|\mu_w|\cos(\angle\mu_w)$, $\mu_y=|\mu_w|\sin(\angle\mu_w)$ yields

\begin{equation}
    \label{eq:AL3}
    \begin{aligned}
    &f_{R,\Theta}(r,\theta) \\
    &{=}\frac{r}{2\pi \sigma^2}\exp\bigg({-}\frac{r^2+|\mu_w|^2}{2\sigma^2}\bigg)\exp\bigg(\frac{r|\mu_w|}{\sigma^2}\cos(\theta{-}\angle\mu_w)\bigg).
    \end{aligned}
\end{equation}
Note that $\theta=\arg\{W(k)\}$. Thus, we turn our attention to mar\-ginal PDF of $\theta$, 

\begin{equation}
    \label{eq:AL4}
    \begin{aligned}
    &f_\Theta(\theta) {=}\int_{0}^{\infty}\frac{r}{2\pi  \sigma^2}\exp\bigg({-}\frac{r^2{-}2r|\mu_w|\cos(\theta{-}\angle\mu_w){+}|\mu_w|^2}{2\sigma^2}\bigg)dr \\
    &{=}\int_{0}^{\infty}\frac{r}{2\pi\sigma^2} \cdot\\
    &\exp\bigg({-}\frac{(r{-}|\mu_w|\cos(\theta{-}\angle{\mu_w}))^2{+}|\mu_w|^2(1{-}\cos^2(\theta{-}\angle\mu_w))}{2\sigma^2}\bigg)dr \\
    &{=}\frac{1}{2\pi}\exp\bigg({-}\frac{|\mu_w|^2(1{-}\cos^2(\theta-\angle\mu_w))}{2\sigma^2}\bigg) \cdot \\
    &\int_{0}^{\infty}\frac{r}{\sigma^2}\exp\bigg({-}\frac{(r{-}|\mu_w|\cos(\theta{-}\angle{\mu_w}))^2}{2\sigma^
    2}\bigg)dr.
    \end{aligned}
\end{equation}
By assuming

\begin{equation*}
\begin{aligned}
    &\alpha{=}|\mu_w|\sin(\theta{-}\angle\mu_w) \\
    &\beta{=}|\mu_w|\cos(\theta{-}\angle\mu_w) \\
    &u{=}\frac{r{-}|\mu_w|\cos(\theta{-}\angle\mu_w)}{\sigma},
    \end{aligned}
\end{equation*}
\eqref{eq:AL4} can be simplified as

\begin{equation}
    \label{eq:AL5}
    \begin{aligned}
    f_\Theta(\theta) &{=}\frac{1}{2\pi}\exp\bigg({-}\frac{\alpha^2}{2\sigma^2}\bigg)\int_{{-}\frac{\beta}{\alpha}}^{\infty}\bigg(u{+}\frac{\beta}{\alpha}\bigg)\exp\bigg({-}\frac{u^2}{2}\bigg)du \\
    &{=}\frac{1}{2\pi}\exp\bigg({-}\frac{\alpha^2}{2\sigma^2}\bigg)\bigg(\int_{{-}\frac{\beta}{\alpha}}^{\infty}ue^{{-}\frac{u^2}{2}}du+\int_{{-}\frac{\beta}{\alpha}}^{\infty}\frac{\beta}{\alpha}e^{{-}\frac{u^2}{2}}du\bigg) \\
    &{=}\frac{1}{2\pi}\exp\bigg({-}\frac{\alpha^2{+}\beta^2}{2\sigma^2}\bigg){+}\frac{\beta}{\sqrt{2\pi}\sigma}\exp\bigg({-}\frac{\alpha^2}{2\sigma^2}\bigg)\Q\bigg({-}\frac{\beta}{\alpha}\bigg) \\
    &{=}\frac{1}{2\pi}\exp\bigg({-}\frac{|\mu_w|^2}{2\sigma^2}\bigg){+}\frac{|\mu_w|\cos(\theta{-}\angle\mu_w)}{\sqrt{2\pi}\sigma} \cdot \\
    &\exp\bigg({-}\frac{|\mu_w|^2\sin^2(\theta{-}\angle\mu_w)}{2\sigma^2}\bigg)\Q\bigg({-}\frac{|\mu_w|}{\sigma}\cos(\theta{-}\angle\mu_w)\bigg)
    \end{aligned}
\end{equation}
where $\Q(\cdot)$ is the $\Q$ function and~\eqref{eq:AL5} gives the explicit 
mar\-ginal PDF for $\theta$. Note that $|\mu_w|=(N-k)(\frac{E_s}{T})^2$ 
from~\eqref{eq:ori_pdf_W}. Thus, at relatively high SNR, i.e., $|\mu_w|\gg\sigma$,
~\eqref{eq:AL5} can be approximated by

\begin{equation}
    \label{eq:AL6}
    \begin{aligned} 
    &f_\Theta(\theta) \approx \frac{1}{2\pi}\exp\bigg({-}\frac{|\mu_w|^2}{2\sigma^2}\bigg)\\
    &+\frac{|\mu_w|\cos(\theta{-}\angle\mu_w)}{\sqrt{2\pi}\sigma} 
    \exp\bigg({-}\frac{|\mu_w|^2\sin^2(\theta{-}\angle\mu_w)}{2\sigma^2}\bigg) \cdot \\
    &\bigg(1{-}\frac{\sigma}{\sqrt{2\pi}|\mu_w|\cos(\theta{-}\angle\mu_w)}\exp\bigg({-}\frac{|\mu_w|^2\cos^2(\theta{-}\angle\mu_w)}{2\sigma^2}\bigg)\bigg) \\
    &=\frac{|\mu_w|\cos(\theta{-}\angle\mu_w)}{\sqrt{2\pi}\sigma}\exp\bigg({-}\frac{|\mu_w|^2\sin^2(\theta{-}\angle\mu_w)}{2\sigma^2}\bigg)
    \end{aligned}
\end{equation}
The approximation holds because of the property of $\Q$ function 
$Q(x) \approx\frac{1}{\sqrt{2\pi x}}\exp\big({-}\frac{x^2}{2}\big)$ 
for $x\gg 0$. Then, we only need to look at $f_\Theta(\theta)$ where 
$\theta \approx \angle\mu_w$, i.e.,

\begin{equation}
    \label{eq:AL7}
    f_\Theta(\theta) \approx \frac{|\mu_w|}{\sqrt{2\pi}\sigma}\exp\bigg({-}\frac{|\mu_w|^2}{2\sigma^2}(\theta{-}\angle\mu_w)^2\bigg)
\end{equation}
Thus, $\theta$ is approximately Gaussian with the distribution 
$\theta \sim \n \big(\angle \mu_{W},\frac{\sigma^2}{|\mu_{W}|^2}\big)$, 
which is equivalent to~\eqref{eq:sol_pdf_W}.

\section{Derivation of the CRVB for frequency and phase estimate, 
proof of~\eqref{eq:CRVB_freql} and~\eqref{eq:CRVB_phi}}

\label{BL}

In order to derive the CRVB for joint estimation of phase and frequency 
offset, We define a parameter vector $\bm{\theta}{\triangleq}\begin{bmatrix} \delta&\phi \end{bmatrix}$ 
to include the two parameters that we want to estimate. 
The first step towards the derivation of the CRVB is to compute 
the Fisher Information Matrix (FIM, $\bm{I}(\bm{\theta})$). 
The calculation of $\bm{I}(\bm{\theta})$ is based on log-likelihood 
function ($\ln\Lambda$), which is carried out in \cite[Ch.~4]{VanTrees_vol1}

\begin{equation}
\label{eq:log_likelihood_func}
\ln\Lambda[r(t),\bm{\theta}]{=}\frac{2}{N_{0}}\int_{0}^{T_{0}}r(t)s'^{*}(t,\bm{\theta})dt{-}\frac{1}{N_{0}}\int_{0}^{T_{0}}|s'(t,\bm{\theta})|^{2}dt
\end{equation}
where 

\begin{equation}
\label{eq:log_lik_func_comp}
s'(t,\bm{\theta})=Ae^{j(2\pi\delta t+\phi)}\sum_{i=0}^{L_{0}-1}c_{i}g(t-iT),
\end{equation}
and $T_{0}$ is the observation length. Taking the second derivative with respect to each element of the parameter vector $\theta_{i}$, $\theta_{j}$ yields

\begin{equation}
\begin{aligned}
\label{eq:double_derivative_theta}
&\frac{\partial^2\ln\Lambda}{\partial\theta_{i}\partial\theta_{j}} \\
&{=}\frac{2}{N_{0}}\int_{0}^{T_{0}}r(t)\frac{\partial^2 s'^{*}(t,\bm{\theta})}{\partial \theta_{i}\partial \theta_{j}}dt
-\frac{2}{N_{0}}\int_{0}^{T_{0}}\frac{\partial s'^{*}(t,\bm{\theta})}{\partial \theta_{i}}\frac{\partial s'(t,\bm{\theta})}{\partial \theta_{j}}dt \\
&-\frac{2}{N_{0}}\int_{0}^{T_{0}}s'(t,\bm{\theta})\frac{\partial^2 s'^{*}(t,\bm{\theta})}{\partial \theta_{i}\partial \theta_{j}}dt.
\end{aligned}
\end{equation}
Taking the negative expectation of (\ref{eq:double_derivative_theta}), the elements of FIM are given by

\begin{equation}
\label{eq:FIM}
\bm{I}(\bm{\theta})_{ij}=-\E\left[\frac{\partial^2 \ln\Lambda}{\partial \theta_{i}\partial \theta_{j}}\right]=\frac{2}{N_{0}}\int_{0}^{T_{0}}\frac{\partial s'^{*}(t,\bm{\theta})}{\partial \theta_{i}}\frac{\partial s'(t,\bm{\theta})}{\partial \theta_{j}}dt.
\end{equation}
By plugging $s'(t,\bm{\theta})$ from (\ref{eq:log_lik_func_comp}) into (\ref{eq:FIM}) 
and replacing $\theta_{i}$, $\theta_{j}$ with $\delta,\delta$, 
the first element of FIM ($\bm{I}(\bm{\theta})_{11}$) yields

\begin{equation}
\label{eq:FIM_delta_delta}
\begin{aligned}
\bm{I}(\bm{\theta})_{11}&=\frac{2}{N_{0}}\int_{0}^{T_{0}}\frac{\partial Ae^{j(2\pi\delta t+\phi)}\sum_{i=0}^{L_{0}-1}c_{i}g(t{-}iT)}{\partial \delta} \cdot \\
&\frac{\partial Ae^{-j(2\pi\delta t+\phi)}\sum_{i=0}^{L_{0}-1}c_{i}^{*}g^{*}(t{-}iT)}{\partial \delta}dt \\
&=\frac{2}{N_{0}}\int_{0}^{T_{0}}A^{2}4\pi^{2}t^{2}\left|\sum_{i=0}^{L_{0}-1}c_{i}g(t{-}iT)\right|^{2}dt.
\end{aligned}
\end{equation}
Note that the averaged symbol energy of transmitted signal is calculated by
\begin{equation}
\label{eq:avg_symbol_energy}
\begin{aligned}
E_{s}&=\int_{0}^{T}\left|A\sum_{i=0}^{L_{0}-1}c_{i}g(t{-}iT)\right|^{2}dt \\
&\approx\sum_{k=0}^{M-1}\left|A\sum_{i=0}^{L_{0}-1}c_{i}g(kT_s-iT)\right|^{2}T_{s} \\
&\approx T\left|A\sum_{i=0}^{L_{0}-1}c_{i}g(t-iT)\right|^{2},
\end{aligned}
\end{equation}
or,

\begin{equation}
\label{eq:avg_symbol_energy_deriv}
A^{2}\left|\sum_{i=0}^{L_{0}-1}c_{i}g(t{-}iT)\right|^{2}{\approx}\frac{E_{s}}{T}.
\end{equation}
where $M=\text{int}(T/T_{s})$. The first approximation in~\eqref{eq:avg_symbol_energy} 
is based on Riemann sum theory. $\bm{I}(\bm{\theta})_{11}$ finally results in

\begin{equation}
\label{eq:FIM_delta_delta_result}
\bm{I}(\bm{\theta})_{11}=\frac{8\pi E_{s}T_{0}^2L_{0}}{3N_{0}}.
\end{equation}
Similarly, $\bm{I}(\bm{\theta})_{12}$ can be calculated by plugging $s'(t,\bm{\theta})$ from (\ref{eq:log_lik_func_comp}) into (\ref{eq:FIM}) and replacing $\theta_{i}$, $\theta_{j}$ with $\delta,\phi$, which is given by

\begin{equation}
\label{eq:FIM_delta_phi}
\begin{aligned}
\bm{I}(\bm{\theta})_{12}&=\frac{2}{N_{0}}\int_{0}^{T_{0}}\frac{\partial Ae^{j(2\pi\delta t+\phi)}\sum_{i=0}^{L_{0}-1}c_{i}g(t{-}iT)}{\partial \delta} \cdot \\
&\frac{\partial Ae^{-j(2\pi\delta t+\phi)}\sum_{i=0}^{L_{0}-1}c_{i}^{*}g^{*}(t{-}iT)}{\partial \phi}dt.
\end{aligned}
\end{equation}
Following the same steps as deriving $\bm{I}(\bm{\theta})_{11}$, $\bm{I}(\bm{\theta})_{12}$ can be finally reduced to
\begin{equation}
\label{eq:FIM_delta_phi_result}
\bm{I}(\bm{\theta})_{12}=\frac{2\pi E_{s}T_{0}L_{0}}{N_{0}}.
\end{equation}
$\bm{I}(\bm{\theta})_{22}$ can be calculated by plugging $s'(t,\bm{\theta})$ from (\ref{eq:log_lik_func_comp}) into (\ref{eq:FIM}) and replacing $\theta_{i}$, $\theta_{j}$ with $\phi,\phi$, which is given by

\begin{equation}
\label{eq:FIM_phi_phi}
\begin{aligned}
\bm{I}(\bm{\theta})_{22}&=\frac{2}{N_{0}}\int_{0}^{T_{0}}\frac{\partial Ae^{j(2\pi\delta t+\phi)}\sum_{i=0}^{L_{0}-1}c_{i}g(t{-}iT)}{\partial \phi} \cdot \\
&\frac{\partial Ae^{-j(2\pi\delta t+\phi)}\sum_{i=0}^{L_{0}-1}c_{i}^{*}g^{*}(t{-}iT)}{\partial \phi}dt.
\end{aligned}
\end{equation}
$\bm{I}(\bm{\theta})_{22}$ can be finally reduced to

\begin{equation}
\label{eq:FIM_phi_phi_result}
\bm{I}(\bm{\theta})_{22}=\frac{2E_{s}L_{0}}{N_{0}}.
\end{equation}
Then, $\bm{I}(\bm{\theta})_{11}$, $\bm{I}(\bm{\theta})_{12}$ and 
$\bm{I}(\bm{\theta})_{22}$ form the FIM

\begin{equation}
\label{eq:FIM_result}
\bm{I}(\bm{\theta})=
\begin{bmatrix}
\frac{8\pi E_{s}T_{0}^2L_{0}}{3N_{0}} & \frac{2\pi E_{s}T_{0}L_{0}}{N_{0}} \\
\frac{2\pi E_{s}T_{0}L_{0}}{N_{0}} & \frac{2E_{s}L_{0}}{N_{0}} \\
\end{bmatrix},
\end{equation}
and the inverse FIM is given by
\begin{equation}
\label{eq:inverse_FIM_result}
\bm{I}^{-1}(\bm{\theta})=
\begin{bmatrix}
\frac{3}{2\pi^{2}L_{0}T_{0}^2E_{s}/N_{0}} & \frac{-3}{2\pi L_{0}T_{0}E_{s}/N_{0}} \\
\frac{-3}{2\pi L_{0}T_{0}E_{s}/N_{0}} & \frac{2}{L_{0}E_{s}/N_{0}} \\
\end{bmatrix}.
\end{equation}
Thus, the CRVB for the frequency and phase estimates are

\begin{equation}
\label{eq:derivation_CRVB_from_FIM}
\begin{aligned}
&\text{CRVB}(\delta) \geq \frac{3}{2\pi^{2}L_{0}T_{0}^{2}E_s/N_{0}} \\
&\text{CRVB}(\phi) \geq \frac{2}{L_{0}E_s/N_{0}},
\end{aligned}
\end{equation}
which are equivalent to~\eqref{eq:CRVB_freql} and~\eqref{eq:CRVB_phi} respectively by replacing $T_0$ with $N$.

\end{appendices}
    

\bibliographystyle{ieeetr}
\bibliography{ref.bib}

\end{document}


